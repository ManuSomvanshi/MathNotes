\section{Lebersgue Measure}
In this section we define a $\sigma-$additive measure on a class of subsets $\R$ and formalise the notion of length of subsets of $\R$. The procedure to do this is as follows:
\begin{enumerate}
  \item Construct a semi-algebra $ \mathscr{S}$ and a $\sigma-$additive measure $\mu: \mathscr{S}\to \R^+$.
  \item Use theorem \cref{thm:extension} to construct a $\sigma$-additive measure $\nu$ on the algebra $ \mathscr{A}$ generated by $ \mathscr{S}$.
  \item Use the caratheodory theorem to determine the $\sigma-$measure on $ \mathscr{F}$, the $\sigma-$algebra generated by $ \mathscr{A}$.
\end{enumerate}
\begin{definition}  
Let $ \mathscr{S} = \{ \emptyset,\ \R,\ (a,b],\ (a,\infty),\ (-\infty, b]\}$. It is easy to check that $ \mathscr{S}$ is a semi-algebra of subsets of $\R$. Let $F:\R\to\R$ be a non-decreasing function. Then define $\mu_F: \mathscr{S}\to \R^+$ as:
\begin{align*}
  \mu_F(\emptyset) = 0,\ \mu_F(\R)=F(\infty),\ \mu_F((a,b]) = F(b)-F(a),\ \mu_F((a,\infty))=F(\infty),\ \text{and}\ \mu_F((-\infty,b]) = F(\infty)
\end{align*}
\end{definition}
\begin{remark}
  Observe that if we construct a function $G:\R\to\R$ given by $G(x) = \lim_{n\to\infty} F(x_n)$, where $x_n \downarrow x$. It is easy to verify that $G$ is non-decreasing, right continuous and $\mu_G(A) = \mu_F(A),\ \forall\ A \in \mathscr{S}$. Hence without loss of generalization it is fair to assume that $F$ is always right continuous.
\end{remark}
\begin{proposition}
  $\mu$ is a $\sigma$-additive measure.
\end{proposition}
\begin{proof}
  By definition we have that $\mu_F(\emptyset) = 0$. 
\end{proof}
