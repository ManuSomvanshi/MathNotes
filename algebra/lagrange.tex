\section{Lagrange's Theorem}
\begin{definition}
  Let $(G,\cdot)$ be a group and $H\subset G$ be a subgroup then define an equivalence relation $\sim$ on $G$ as follows: $a\sim b \iff ab^{-1}\in H$.
\end{definition}
\begin{definition}
  Let $(G,\cdot)$ be a group, and $H$ be a subgroup of $G$ then define $Ha = \{ga\ |\ g\in H\}$, where $a\in G$. $H$ is said to be a right coset in $G$.  
\end{definition}
\begin{proposition}
  The equivalence class $[a]$ w.r.t. the equivalence relation $\sim$ on $(G,\cdot)$ is the as the set $Ha$. 
\end{proposition}
\begin{proof}
  If $b\in [a]$ then
  \begin{align*}
    ab^{-1} &= g\ (\in H),\\
    \implies b &= g^{-1}a\ \text{where}\ g^{-1}\in H\ \text{since $H$ is a group},\\ 
    \implies b &\in Ha \implies [a]\subset Ha.
  \end{align*}
  On the other hand if $b\in Ha$ then,
  \begin{align*}
    b &= ga\\
    \implies g^{-1} &= ab^{-1}\in H,\ \text{again since $H$ is a subgroup}\\
    \implies b &\in [a] \implies Ha \subset [a].
  \end{align*}
  This completes the proof.
\end{proof}
\begin{theorem}[Lagrange's Theorem]
  If $G$ is a finite group and $H\subset G$ is a subgroup then the order of $H$ divides order of $G$.
\end{theorem}
\begin{proof}
  We consider the equivalence classes defined above. Let's say there are $k$ distinct equivalence classes. Then $G = \cup_{j=1}^{n} Ha_k$, and $Ha_i \cap Ha_j = \emptyset$ if $i\neq j$ (since equivalence classes form a partition of the set). Let $f_{a}: H \to Ha$ be given by $f_a(g) = ga$. If $f_a(g) = f_a(h)$ then
  \begin{align*}
    ga &= ha\\
    \implies g &= h
  \end{align*}
  Thus $f_a$ is injective. Let $h\in Ha$, then $\exists\ g\in H$ such that $h = ga$. Thus $ha^{-1} = g \in H$. Hence for any $h\in Ha$ it is possible to find a $g\in H$ (given by $ha^{-1}$) such that $h = f_a(g)$. Therefore $f_a$ is surjective as well, making $f_a$ a bijection. This means that $|H| = |Ha|$. Since each $Ha_j$ is disjoint the union has $k|H|$ elements. Thus $|G| = k|H|$.
\end{proof}
\begin{remark}
  The number of right cosets $H$ in $G$ is called the index of $H$ in $G$, denoted $i_G(H)$. As seen in the proof of Lagrange's theorem $i_G(H) = |G|/|H|$.
\end{remark}
\begin{corollary}
  Every group $G$ of prime order, $p$, is cyclic.
\end{corollary}
\begin{proof}
  From Lagrange's theorem any subgroup $H$ of $G$ can either be $\{e\}$ or $G$ since only $1,p$ divide $p$. If $H=G$ and $a(\neq e)\in G$ then $ \langle a \rangle$ forms a subgroup of $G$ different from $\{e\}$. Thus $ \langle a \rangle = G$, proving that $G$ is cyclic.
\end{proof}
\begin{corollary}
  If $a\in G$, where $G$ is a finite group then $ \mathcal{O}(a)$ divides $|G|$.
\end{corollary}
\begin{proof}
  Since $ \mathcal{O}(a)$ is the cardinality of the subgroup generated by $a$, then by Lagrange's theorem it divides $|G|$.
\end{proof}
\begin{corollary}
  If $G$ is a finite group then $a^{|G|} = e$ for all $a\in G$.
\end{corollary}
\begin{proof}
  Let $a\in G$ with order $k$. From Lagrange's theorem $|G| = mk$. Then $a^{|G|} = (a^{k})^m = e^m = e$.  
\end{proof}
\begin{theorem}[Euler]
  If $a$ is relatively prime to $n$ then $a^{\phi(n)}\bmod(n) = 1$, where $\phi(n)$ is the Euler totient function (defined as the number of coprimes of $n$).
\end{theorem}
\begin{proof}
  Consider the group $((\Z\bign/n)^*, \times))$. We have already seen that it has a cardinality $\phi(n)$. Then by the above corollary $(a\bmod(n))^{\phi(n)} = a^{\phi(n)}\bmod(n) = 1$.
\end{proof}
\begin{corollary}[Fermat's Little Theorem]
  If $p$ is a prime and $p$ does not divide $a$ then $a^{p-1} \bmod(p) = 1$.
\end{corollary}
\begin{proof}
  In Euler's theorem consider the case when $n=p$. Then the cardinality of the group $((\Z\bign/p),\times)$ is $p-1$. Thus $a^{p-1}\bmod(n) = 1$.
\end{proof}
