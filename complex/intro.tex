\section{Algebra of Complex Numbers}
\begin{definition}
  Let $\C = \R \times \R$. Define addition and multiplication on $\C$ in the following way:
  \begin{align*}
    (x,y) + (a,b) &= (x+a,y+b) \\
    (x,y)(a,b) &= (xa-yb, ay+bx),
  \end{align*}
  where addition and multiplication of $\R$ is the usual one. $\C, +, \cdot$ is called the complex numbers.
\end{definition}
\begin{remark}
  It can be shown easily that $\C$ is a field with this addition and multiplication with additive identity $(0,0)$ and multiplicative identity $(1,0)$.
\end{remark}
\begin{proposition}
  The field of real numbers is isomorphic to a subfield of $\C$. 
\end{proposition}
\begin{proof}
  Consider the map $a\mapsto (a,0)$. It is easy to see that addition and multiplication are preserved. Clearly the map is also a bijection. Thus it is a field isomorphism to the subfield $\{(a,0)\ |\ a\in \R\}$.
\end{proof}
\begin{notation}
  From here on if I say some real $a$ belongs to $\C$, I mean the element $(a,0)$.
\end{notation}
\begin{definition}
  Let $i = (0,1)\in \C$.
\end{definition}
\begin{theorem}
  $i^2 = -1$. 
\end{theorem}
\begin{proof}
  By definition of the product:
  \begin{align*}
    (0,1)(0,1) = (-1, 0).
  \end{align*}
\end{proof}
\begin{notation}
  The element $(a,b)$ would be from here on represented as $a+ib$.
\end{notation}
\begin{theorem}
  $\C$ is not an ordered field. 
\end{theorem}
\begin{proof}
  Suppose that $<$ is a total ordering on the field $\C$. Either $i<0$, $i>0$, or $i=0$. Clearly the last case is not possible. Suppose that $i<0$. Then $-i>0$ and therefore by properties of ordered field $(-i)(-i)=-1>0$. Again using the same property $-1\times -1 =1 > 0$. But since $-1>0$ we must also have $1<0$. This is a contradiction. Thus $i<0$ is not possible. Similarly $i>0$ is not possible. Thus $\C$ cannot be ordered.
\end{proof}
\begin{definition}
  Given a complex number $z = a+ib$ define the compliment of $z$, $\bar{z} = a-ib$.
\end{definition}
\begin{theorem}
  The map $z\mapsto \bar{z}$ is a field isomorphism.
\end{theorem}
\begin{proof}
  Easy to check.
\end{proof}
\begin{remark}
  This shows that $z$ is indistinguishible from $\bar{z}$. Given any equation in terms of $z$, replacing $z$ with $\bar{z}$ every makes no difference.
\end{remark}
\begin{theorem}
  If $z,w\in \C$ then
   \begin{enumerate}
     \item $\overline{z+w} = \bar{z} + \bar{w}$,
     \item $\overline{zw} =\bar{z}\bar{w}$, 
     \item $z\bar{z}$ is real and positive.
   \end{enumerate} 
\end{theorem}
\begin{proof}
  Can be easily show by writting $z=a+ib$ and $w=x+iy$.
\end{proof}
\begin{definition}
  Define $|z| = \sqrt{z\bar{z}}$.
\end{definition}
\begin{remark}
  It is easy to check that $|z|$ defines a norm on $\C$.
\end{remark}
