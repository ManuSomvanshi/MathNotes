\section{Ideals}
\begin{definition}
  Let $R$ be a ring and let $S$ be a non-empty subset which is closed under addition and multiplication in $R$. If $S$ itself is a ring under these operations then its called a subring.
\end{definition}
\begin{definition}
  If $S$ is a subring of $R$ and
  \begin{align*}
    \forall r\in R\ s\in S \implies rs \in S,
  \end{align*}
  then $S$ is called a left ideal. Similarly we define a right ideal. If $S$ is both left and right ideal then it's called an ideal.
\end{definition}
\begin{example}
  Let $R$ be any ring. Then the center $C = \{x\in R\ |\ rx = xr,\ \forall r\in R\}$ is a subring. But it may not always be an ideal.
\end{example}
\begin{example}
  Let $f: R\to S$ be a homomorphism of rings. Then $\Ker f$ is an ideal. This can be easily checked: if $x\in \Ker f$ then $f(x) = 0$, if $r\in R$ then $f(rx) = f(r)f(x) = 0$. Similarly for right multiplication. The image of $f$ is also an ideal in $S$.
\end{example}
\begin{remark}
  An ideal $I$ of a ring $R$ is said to be proper or non-trivial if $I\neq \{0\},R$. Observe that if $R$ has an identity then $I=R$ if and only if $1_R\in I$. Also note that any theorem proved for left ideals similarly applies to a right ideal.
\end{remark}
\begin{theorem}
  Any non-empty subset $I$ of a ring $R$ is a left ideal of if and only if for all:
  \begin{align*}
    a,b\in I \implies a-b\in I \And a\in I, r\in R \implies ra = I.
  \end{align*}
\end{theorem}
\begin{proof}
  Supposing $I$ is an ideal then using group properties $a-b\in I$ and by definition of ideals $ra\in I$. Conversely suppose that $a-b\in I$ for all $a,b\in I$. This shows that every $b\in I$ has inverse in $I$, and $0\in I$, and that $I$ is closed. Thus it follows that $I$ is a subring. By definition of left ideal we can see that $I$ is a left ideal due to the second property.
\end{proof}
\begin{corollary}
  If $\{A_\a\ |\ \a\in I\}$ is a collection of left ideals of $R$ then $A = \bigcap_\a A_\a$ is also a left ideal.
\end{corollary}
\begin{proof}
  Suppose that $a,b\in A$. Then $b\in A_\a$ for all $\a$. Thus $-b\in A_\a \implies -b\in A$. Again using the same argument $a-b\in A$. Let $r\in R$. Then $ra\in A_\a$ for each $\a$. Therefore $ra\in A$, which means that $A$ is a left ideal. 
\end{proof}
\begin{definition}
  Let $X$ be any non-empty subset of ring $R$. Let $X_\a$ be the left ideals that contain $X$ then $\bigcup_\a X_\a$ is called the left ideal generated by $X$.
\end{definition}
\begin{definition}
  An ideal generated by a single element is called a principle ideal. A ring in which all the ideals are principle ideals is called a principle ideal ring. If a principle ideal ring is an integral domain then its called a principle ideal domain.
\end{definition}
\begin{notation}
  A ideal generated by $\{x_1,\cdots, x_n\}$ is written as $(x_1,\cdots, x_n)$. The ideal generated by the set $X$ is denoted $(X)$.
\end{notation}
\begin{theorem}
  Let $R$ be a ring, $a\in R$, and $X\subset R$ then:
  \begin{enumerate}
    \item The principle ideal $(a)$ consists of  all elements of the form $ra+ as + na + \sum_{i=1}^m r_i a s_i$ where $r,r_i,s,s_i\in R$ and $n,m\in \N$.
    \item If $R$ has an identity then $(a) = \{\sum_{i=1}^m r_i a s_i\ |\ r_i,s_i \in R, m\in N\}$.
    \item If $a$ is in the center of $R$ then $(a) = \{ra+na\ |\ r\in R,\ n\in \N\}$.
    \item $Ra = \{ra\ |\ r\in R\}$ is a left ideal in $R$.
    \item If $R$ has an identity and $a$ is in the center of $R$ then $Ra = (a) = aR$.
    \item If $R$ has an identity and $X$ is the center of $R$ then the ideal $(X)$ consists of all finite sums $\sum_{i=1}^n r_ia_i$ where $r_i\in R,\ a_i\in X$.
  \end{enumerate}
\end{theorem}
\begin{proof}[proof sketch]
  \begin{enumerate}
    \item Consider the set $I = \{ra+ as + na + \sum_{i=1}^m r_i a s_i\}$. It is easy to show that this is an ideal. Let $K$ be any ideal of $R$ which contains $a$. Then $ra\in K,\ na\in K, as\in K, r_ias_i\in K$ by definition of ideals. Thus it follows that $I\subset K$ for any $K$ containing $a$. Hence $I = (a)$.
    \item Suppose that $R$ has identity. Then $ra = ra1_R$, $as = 1_R as$, and $na = (n1_R)a 1_R$. Thus every element can be wriiten in the form $r_ias_i$.
    \item If $a$ is in the center then $as = sa$ and $ras = (rs)a$. Thus every element can be written in the form $ra+na$.
    \item If $x\in Ra$ then there exists $r$ such that $x = ra$. For any $r'\in R$, $r'x = r'rx = r'' x$ for some $r''\in R$. Thus $Ra$ is a left ideal.
    \item If $a$ is in the center in $R$ then clearly $Ra = aR$. Since $R$ has an indentity and $a$ is in the center, by parts 2 and 3 we have that $(a) = \{ra\ |\ r\in R\} = Ra = aR$.
    \item Suppose that $I = \{\sum_{i=1}^n r_ia_i\ |\ r_i\in R\}$. Since $r'(ra) = (r'r)a = r''a$ and $rar' = (rr')a$ (since $X$ is in the center). Thus $I$ is an ideal. Any ideal $K$ containing $X$ contains $I$ since $a_i\in K \implies r_i a_i\in K$ and so is their sum. Thus $I = (X)$.
  \end{enumerate}
\end{proof}
\begin{definition}
  If $A_1,\cdots, A_n$ are non-empty subsets of ring $R$ then denote $A_1 + \cdots + A_n = \{a_1+\cdots + a_n\ |\ a_i\in A_i\}$. If $A,B$ are non-empty then let $AB$ denote $\{a_1b_1+\cdots +a_nb_n\ |\ a_i\in A,\ b_i\in B\}$.
\end{definition}
\begin{theorem}
  Let $A_1,\cdots, A_n$, $B$, and $C$ be left ideals of a ring $R$.
  \begin{enumerate}
    \item $A_1+\cdots + A_n$ and $A_1\cdots A_n$ are left ideals.
    \item $A+(B+C) = (A+B)+C$.
    \item $A(BC) = (AB)C$.
    \item $B(A_1+\cdots + A_n) = BA_1 + \cdots + BA_n$.
  \end{enumerate}
\end{theorem}
\begin{proof}[proof sketch]
  meh... its easy.
\end{proof}
\begin{remark}
  Thus given an ideal $I$ of a ring $R$, the quotient $R\bign/ I$ can be seen as ring as a consequence of the above theorem. The elements of $R/I$ will be written as $a+I$ (following the group theory notation) where $a\in R$. Clearly $I$ is a normal subgroup of $R$ under the operation $+$. Also it can be seen that it is closed under the multiplication:
  \begin{align*}
    (a+I)(b+I) = ab + aI + bI + II = ab + I + I + I = ab +I.
  \end{align*}
  and the product is distributive and associative as a consequence of the above theorem. The additive identity of this ring is $I$. If $R$ has an identity $1_R$ then the multiplicative identity of $R\bign/I$ will be $1_R+I$.
\end{remark}
\begin{theorem}
  If $I$ is an ideal of $R$ then there exists a ring homomorphism $\pi:R\to R\bign/ I$ with kernel $I$.
\end{theorem}
\begin{proof}
  Let the map be given by $a \mapsto a+I$. Then:
  \begin{align*}
    \pi(ab) = ab +I = (a+I)(b+I) = \p(a)\p(b)\\
    \pi(a+b) = a+b+I = a+I + b+I = \pi(a)+\pi(b).
  \end{align*}
  The kernel is given by $\Ker \pi = \{a\in R\ |\ a+I = I\}$. This only happens when $a\in I$ due to $I$ being closed under addition. Thus $\Ker \pi = I$.
\end{proof}
\begin{theorem}[First Isomorphism Theorem]
  Let $f:R\to S$ be a homomorphism. Then $R\bign/\Ker f \simeq S$.
\end{theorem}
\begin{proof}[proof sketch]
  The map $\bar{f}: R\bign/ \Ker f \to S$ given by $\bar{f}(a+I) = a$ does the job.
\end{proof}
\begin{theorem}[Second and Third isomorphism theorems]
  If $I$ and $J$ are ideals of a ring $R$ then
  \begin{enumerate}
    \item There is an isomorphism of rings $I\bign/I\cap J \simeq (I+J)\bign/J$.
    \item If $I\subset J$ then $J\bign/I$ is a ring in $R\bign/I$ and there is an isomorphism $(R\bign/ I)\bigg/ (R\bign/J) \simeq R\bign /J$.
  \end{enumerate}
\end{theorem}
\begin{proof}
  Proof is similar to that of the group isomorphism theorems.
\end{proof}
\begin{theorem}
  Every ideal of the ring $R\bign/ I$ is of the form $J\bign/I$ where $J$ is an ideal of $R$ containing $I$.
\end{theorem}
\begin{proof}
  Again similar to that of the groups one.
\end{proof}
\begin{definition}
  An ideal $P\subset R$ is said to be a prime ideal if for any two ideals $A,B$ of $R$,
  \begin{align*}
    AB \subset P \implies A\subset P \qq{or} B\subset P
  \end{align*}
\end{definition}
\begin{theorem}
  If $P$ is an ideal of $R$ such that for all $a,b\in R$
  \begin{align*}
    ab\in P \implies a\in P \qq{or} b\in P,
  \end{align*}
  then $P$ is prime. Conversly if $P$ is prime and $R$ is commutative then $ab\in P \implies a\in P$ or $b\in P$.
\end{theorem}
\begin{proof}
  Suppose that $A,B$ are ideals such that $AB\subset P$ and $A$ is not a subset of $P$ (if it is then we are done). Let $a\in A-P$. For any $b\in B$, we know that $ab\in P$ (since $AB \subset P$) then either $a\in P$ or $b\in P$. But since $a\in A-P$ we cannot have $a\in P$. Thus $b\in P$ for every $b\in B$. Hence $B\subset P$.\\

  Consider $a,b\in R$ such that $ab\in P$. This means that the ideal $(ab) \subset P$. Consider the ideals $(a)$ and $(b)$. We know that $x\in (a)$, $y\in (b)$ means that $x = ra + na$ and $y = sb + mb$ (this is because $R$ is commutative and thus all elements are in the center of $R$). It can be easily checked that $(a)(b) \subset (ab) \subset P$. Since $P$ is prime either $(a)\subset P$ or $(b)\subset P$. Thus either $a\in P$ or $b\in P$.
\end{proof}
\begin{theorem}
  In a commutative ring with identity an ideal $P$ is prime if and only if $R\bign/ P$ is an integral domain.
\end{theorem}
\begin{proof}
  Suppose that $P$ is prime. Let $a+P, b+P\in R\bign/P$. Then we get that
  \begin{align*}
    (a+P) (b+P) = P \implies ab + P = P \implies ab\in P \implies a\in P\ \text{or}\ b\in P \implies a+P = P\ \text{or}\ b+P = P.
  \end{align*}
  This means that $R\bign/P$ is an integral domain.\\

  Conversly suppose that $R\bign/P$ is integral domain. Let $a,b\in R$ such that $ab\in P$. Then
  \begin{align*}
    ab + P = P \implies (a+P)(b+P) = P \implies a+P = P\ \text{or}\ b+P\in P\implies a\in P\ \text{or}\ b\in P.
  \end{align*}
  Thus $P$ is prime.
\end{proof}
\begin{definition}
  A left ideal $M$ in a ring $R$ is said to be a maximal left ideal if $M\neq R$ and for every left ideal $N$ such that $M\subset N \subset R$ either $N=R$ or $N=M$. 
\end{definition}
\begin{theorem}
  In a non-zero ring $R$ with identity, a maximal left ideal always exists. In fact every left ideal in $R$, except $R$, is contained in the maximal ideal.
\end{theorem}
\begin{proof}
  Since $0$ is an ideal it follows that there exists at least one ideal of $R$. By Zorn's lemma we prove that maximal ideal exists.
\end{proof}
