\section{Topological Spaces and Continuous Functions}
\begin{definition}
  A topological space is a pair $(X, \mathscr{T})$ where $X$ is a set and $ \mathscr{T}\subset \mathfrak{P} (X)$ such that:
  \begin{enumerate}
    \item Both $X$ and $\emptyset$ are in $ \mathscr{T}$.
    \item The union of elements of any subset of $ \mathscr{T}$ is in $ \mathscr{T}$.
    \item The intersection of elements of any finite subset of $ \mathscr{T}$ is in $ \mathscr{T}$.
  \end{enumerate}
  $ \mathscr{T}$ is said to be a topology on the set $X$.
\end{definition}
\begin{remark}[Abuse of Notation]
  By definition a topological space is $(X, \mathscr{T})$, but for convinience of typing I will generally just say $X$ is a topological space given that no confusion arises.
\end{remark}
\begin{definition}
  An open set of a topological space $X$ is a set which belongs to the topology $ \mathscr{T}$.
\end{definition}
\begin{definition}
  If $(X, \mathscr{T})$ and $(X, \mathscr{T}')$ are topological spaces such that $ \mathscr{T} \subset \mathscr{T}'$ then $ \mathscr{T}'$ is said to be a finer than $ \mathscr{T}$. In case $ \mathscr{T} \subsetneq \mathscr{T}'$ then $ \mathscr{T}'$ is said to be strictly finer. Also $ \mathscr{T}$ may be said to coarser than or strictly coarser than $ \mathscr{T}'$ respectively in the above two cases. $ \mathscr{T}$ is comparable to $ \mathscr{T'}$ if $ \mathscr{T} = \mathscr{T}'$.
\end{definition}
Generally it is very hard to completely specify every element in the topology. Thus we use the concept of a basis and specify the topology in terms of these basis.
\begin{definition}
  A basis of a topology on $X$ is a collection, $ \mathscr{B}$, of subsets of $X$ such that:
  \begin{enumerate}
    \item $\forall x\in X \ \exists B\in \mathscr{B} (x\in B)$.
    \item $\forall B_1,B_2 \in \mathscr{B}\ \forall x\in B_1\cap B_2\exists B_3 \in \mathscr{B} (x\in B_3 \land  B_3 \subset B_1\cap B_2)$. (This can be extended to finitely many sets $B_1,B_2,\cdots, B_k$).
  \end{enumerate}
\end{definition}
\begin{definition}\label{def:gen_top}
  The topology $ \mathscr{T}$ generated by a basis $ \mathscr{B}$ is defined as follows: A set $U$ is open if for all $x\in U$ there exists a $B\in \mathscr{B}$ such that $x\in B$ and $B \subset U$.
\end{definition}
\begin{proposition}
  The topology $ \mathscr{T}$ in \cref{def:gen_top} is indeed a topology.
\end{proposition}
\begin{proof}
  Clearly $\emptyset \in \mathscr{T}$ since the statement is vacuously true. Also $X\in \mathscr{T}$ since by definition of basis for each $x$ there exists a $B\in \mathscr{B}$ containing $x$ and obviously $B\subset X$ (since $X$ is the whole space). Suppose $\{U_\a\}$ is some collection of open sets of $ \mathscr{T}$. Let $U = \bigcup_\a U_\a$. If $x\in U$ then $x\in U_\a$ for some $\a$. Since $U_\a$ is open it follows that there exists a $B\in \mathscr{B}$ such that $x\in B$ and $B\subset U_\a$. Since $U_\a\subset U$ it follows that $B\subset U$. Thus $U\in \mathscr{T}$. Suppose $U_1,\cdots , U_n \in \mathscr{T}$. Let $U = \bigcap_{k=1}^n U_k$. If $x\in U$ then $x\in U_k$ for all $k$. Since each of them are open there exists $B_k \in \mathscr{B}$ s.t. $x\in B_k$ and $B_k \subset U_k$ for each $k$. One can find a $B\in \mathscr{B}$ such that $B\subset \bigcap_{k=1}^n B_k$. Clearly $x \in B$ and $B\subset \bigcap_{k=1}^n B_k \subset U$. Thus $U\in \mathscr{T}$. This shows that $ \mathscr{T}$ is a topology.
\end{proof}
\begin{lemma}
  Suppose $ \mathscr{B}$ is the basis of the topology $ \mathscr{T}$. Then $ \mathscr{T}$ is the collection of all possible unions of sets in $ \mathscr{B}$.
\end{lemma}
\begin{proof}
  Since $ \mathscr{B} \subset \mathscr{T}$ and $ \mathscr{T}$ is a topology, it follows that all possible unions of $ \mathscr{B}$ are contained in $ \mathscr{T}$. Now suppose that $U\in \mathscr{T}$. Then by definition of generated topology it is possible to find a $B_x\in \mathscr{B}$ for each $x\in U$ such that $B_x \subset U$. Clearly $\bigcup_{x\in U} B_x = U$. Therefore $ \mathscr{T}$ is exactly the collection of all possible unions of sets in $ \mathscr{B}$.
\end{proof}
\begin{lemma}
  Suppose that $(X, \mathscr{T})$ is a topological space. The collections $ \mathscr{C}$ of open sets such that for each open set $U$ of $X$ and for each $x\in U$ there is a $C\in \mathscr{C}$ such that $x\in C \subset U$. Then $ \mathscr{C}$ is a basis of $ \mathscr{T}$.
\end{lemma}
\begin{proof}
  Suppose that $x\in X$. Then by the hypothesis of the lemma there exists $C\in \mathscr{C}$ such that $x\in C$. Suppose that $C_1,C_2\in \mathscr{C}$ such that $x\in C_1\cap C_2$. Since $C_1, C_2$ are open sets it follows that $C_1\cap C_2$ is also open. Thus by definition of $ \mathscr{C}$, there exists a $C_3\in \mathscr{C}$ such that $x\in C_3$ and $C_3\subset C_1 \cap C_2$. Therefore $ \mathscr{C}$ is a basis for $ \mathscr{T}$.
\end{proof}
\begin{lemma}
  Let $ \mathscr{B}$ and $ \mathscr{B}'$ be basis for topologies $ \mathscr{T}$ and $ \mathscr{T}'$ respectively. Then the following are equivalent
  \begin{enumerate}
    \item $\mathscr{T}'$ is finer than $ \mathscr{T}$.
    \item For all $x\in X$ and for all $B\in \mathscr{B}$ with $x\in B$ there exists a $B'\in \mathscr{B}'$ such that $x\in B' \subset B$.
  \end{enumerate}
\end{lemma}
\begin{proof}
  Suppose that $ \mathscr{T}'$ is finer than $ \mathscr{T}$. Then $ \mathscr{T} \subset \mathscr{T}'$. Suppose that $B\in \mathscr{B}$ and $x\in B$ for some $x\in X$ then $B\in \mathscr{T}$ and therefore $B\in \mathscr{T}'$. This means that $B = \bigcup_{\a}B'_\a$ where $B'_\a\in \mathscr{B}'$. Since $x\in B$ it means that $x\in B'_\a$ for some $\a$. Therefore there exists $B_\a\in \mathscr{B}'$ such that $x\in B'_\a$ and $B'_\a \subset B$.\\

  Now suppose the converse. Let $U\in \mathscr{T}$. Since for all $x\in U$ there exists $B\in \mathscr{B}$ such that $x\in B \subset \mathscr{B}$. Since, by assumption, there exists $B'\in \mathscr{B}'$ such that $x\in B' \subset B$ it follows that $x\in B' \subset U$. Thus $U$ is open w.r.t $ \mathscr{T}'$. Hence $ \mathscr{T} \subset \mathscr{T}'$.
\end{proof}
\begin{definition}
  If $ \mathscr{B}$ is the collection of all open intervals of $\R$ then the topology generated by $ \mathscr{B}$ is called the standard topology on $\R$. The topology generated by the collection $ \mathscr{B}'$ of all intervals $[a,b)$ is called the lower limit topology and is represented simply by $\R_\ell$. Let $K = \{1/n\ |\ n\in \Z^+\}$. Then the topology generated by the collection $ \mathscr{B}''$ of all open intervals $(a,b)$ along with the sets $(a,b) - K$ is called the $K$-topology on $\R$ and is written as $\R_K$.
\end{definition}
\begin{lemma}
  The topologies $\R_\ell$ and $\R_K$ are finer than the standard topology $\R$.
\end{lemma}
\begin{proof}
  Suppose that $x\in (a,b) \in \mathscr{B}$. Then the set $[x,b)$ contains $x$ and $[x,b) \in (a,b)$. Since that $[x,b)\in \mathscr{B}''$ but there is no open interval that contains $x$ and is a subset of $[x,b)$. Thus by previous lemma it follows that $\R_\ell$ is strictly finer than $\R$.\\

  For any $(a,b)\in \mathscr{B}$ the same interval is in $ \mathscr{B}''$. Consider $(-1,1) - K \in \mathscr{B}''$. Due to the denseness of $\Q$ in $\R$ there can be no interval $(a,b)$ containg $0$ that is also a subset of $(-1,1)-K$. Thus $\R_K$ is a strictly finer topology than $\R$.
\end{proof}
\begin{definition}
  A subbasis for a topology $ \mathscr{T}$ is a collection of subsets $ \mathscr{S}$ of $X$ such that the union of all elements of $ \mathscr{S}$ is $X$.\\

  The topology generated by a subbasis $ \mathscr{S}$ is defined as the collection of all unions of finite intersections of elements of $ \mathscr{S}$.
\end{definition}
\begin{observation}
  It is easy to see that the topology generated by a subbasis is indeed a topology. This can be shown by just showing that the collection of finite intersections of elements of $ \mathscr{S}$ forms a basis.
\end{observation}

\subsection{Order Topology}
\begin{definition}
  Given an ordered set $(X,<)$, let $ \mathscr{B}$ be the collection containing:
  \begin{enumerate}
    \item All intervals $(a,b) \coloneqq \{x\in X\ |\ a<x<b\}$.
    \item If $X$ has minimum element $a_0$ then the intervals $[a_0, b) \coloneqq \{x\in X\ |\ a_0\leq x<b\}$. 
    \item If $X$ has maximum element $b_0$ then the intervals $(a, b_0] \coloneqq \{x\in X\ |\ a< x\leq b_0\}$. 
  \end{enumerate}
\end{definition}
\begin{observation}
  The collection $ \mathscr{B}$ forms a basis since:
  \begin{itemize}
    \item If $X$ has a minimum element then it belongs to the interval $[a_0, b)$. Similarly if it has a maximum element it belongs to $(a,b_0]$. Any other $x\in X$ would be present in intervals of the form $(a,b)$.
    \item The intersection of intervals would again yield intervals. This can be easily checked.
  \end{itemize}
\end{observation}
\begin{example}
  The most trivial example is order topology on $\R$. This yields the standard topology on $\R$. As a non-trivial example consider $\R\times \R$. Denote an element of $\R\times \R$ as $x\times y$. Consider the order relation defined in the following way: we say $x\times y < x'\times y'$ if $x<x'$ or if $x=x'$ but $y<y'$. Since there is no largest or smallest element the order topology is generated by
  \begin{align*}
    \mathscr{B} = \{(x\times y, x'\times y')\ |\ x<x'\ \lor\ x=x' \And y<y'\}.
  \end{align*}
\end{example}

\subsection{Product Topology}
\begin{definition}
  Given two topological spaces $(X, \mathscr{T}_X)$ and $(Y, \mathscr{T}_Y)$ define the topology on $X\times Y$ as being generated by
  \begin{align*}
    \mathscr{B} = \{U\times V\ |\ U\in \mathscr{T}_X \And V\in \mathscr{T}_Y\}.
  \end{align*}
\end{definition}
\begin{observation}
  Again it is required to prove that $ \mathscr{B}$ is indeed a basis.
\end{observation}
\begin{theorem}
  If $\mathscr{B}$ is a basis for topology of $X$ and $ \mathscr{C}$ is basis for topology of $Y$ then
  \begin{align*}
    \mathscr{D} = \{B\times C\ |\ B\in \mathscr{B} \And C\in \mathscr{C}\}.
  \end{align*}
  is a basis for the product topology.
\end{theorem}
\begin{proof}
  If $(x,y)\in U\times V\subset X\times Y$ then there exists $B\in \mathscr{B}$ and $C\in \mathscr{C}$ such that $x\in B\subset U$ and $y\in C \subset V$. Thus $(x,y)\in B\times C \subset U\times V$. 
\end{proof}
\begin{definition}
  Let $\pi_1:X\times Y \to X$ be $\pi_1(x,y) = x$. Similarly define $\pi_2$ for $Y$.
\end{definition}
\begin{observation}
  Notice that $\pi_i$ are onto maps. Also note that $\pi_1^{-1}(U) = U \times Y$ and $\pi_2^{-1}(V) = X\times V$. Both of these are open in $X\times Y$ if $U,V$ are open resp.
\end{observation}
\begin{theorem}
  The collection
  \begin{align*}
    \mathscr{S} = \{\pi_1^{-1}(U)\ |\ U\in \mathscr{T}_X\} \cup \{\pi_2^{-1}(U)\ |\ V\in \mathscr{T}_Y\}
  \end{align*}
  is a subbasis for the product topology.
\end{theorem}
\begin{proof}
  Let $ \mathscr{T}$ be the product topology and $ \mathscr{T}'$ be the topology generated by $ \mathscr{S}$. Since every element of $ \mathscr{T}'$ is of the form $U\times Y$ or $X\times Y$, it follows that $ \mathscr{S} \subset \mathscr{T}$. Thus by definition of generated subset $ \mathscr{T}' \subset \mathscr{T}$. If $U\times V$ is in the basis which generates the product topology. Since any $U\times V$, $U,V$ open, can be expressed as $\pi_1^{-1}(U) \cap \pi_2^{-1}(V)$ it follows that the basis of the product topology is contained in the basis generated by$ \mathscr{S}$. Thus $ \mathscr{T} \subset \mathscr{T}'$. Therefore $ \mathscr{T} = \mathscr{T}'$.
\end{proof}

\subsection{Subspace Topology}
\begin{definition}
  Given a topological space $(X, \mathscr{T}')$ it we can define a topology on any subset $Y\subset X$ in the following way:
  \begin{align*}
    \mathscr{T}_Y = \{Y\cap U\ |\ U\in \mathscr{T}\}.
  \end{align*}
  This is called the subspace topology.
\end{definition}
\begin{theorem}
  If $ \mathscr{B}$ is a basis of topology for $X$, then $ \mathscr{B}_Y = \{Y\cap B\ |\ B\in \mathscr{B}\}$ is a basis for the subspace topology of $Y\subset X$.
\end{theorem}
\begin{proof}
  Let $U \in \mathscr{T}_X$ and $y\in Y$ such that $y\in U$. Then there exists a $B\in \mathscr{B}$ such that $y\in \mathscr{B} \subset U$. Since $y\in B\cap Y$ it follows that there exists $C\in \mathscr{B}_Y$ such that $y\in C \subset U$. This shows that $ \mathscr{B}_Y$ is a basis for subspace topology.
\end{proof}

\subsection{Closed Sets, Closure, Interior}
\begin{definition}
  A subset $A$ of a topological space $X$ is called closed $A^c$ is open.
\end{definition}
\begin{remark}
  Since $\l(A^c\r)^c = A$ it follows that compliments of open sets are closed.
\end{remark}
\begin{theorem}
  Let $X$ be a topological space then:
  \begin{enumerate}
    \item $\emptyset$ and $X$ are closed,
    \item Arbitrary intersection of closed sets is closed,
    \item Finite union of closed sets is closed.
  \end{enumerate}
\end{theorem}
\begin{proof}
  Since $\emptyset^c = X$ and $X^c = \emptyset$ they are closed. Let $A_\a$ be closed sets. Then $A_\a^c$ open. Since arbitrary union of open sets is open:
  \begin{align*}
    \bigcup_\a A_\a^c = \l(\bigcap_\a A_\a\r)^c
  \end{align*}
  Thus $\bigcup_\a A_\a$ is closed. Similarly it can be shown that finite intersections are closed.
\end{proof}
\begin{theorem}
  If $Y$ is a subspace of $X$ and $A$ is a subset of $Y$. Then $A$ is closed in $Y$ if and only if $A$ can be written as the intersection of a closed set of $X$ and $Y$.
\end{theorem}
\begin{proof}
  Let $A = C\cap Y$ where $C$ is closed in $X$. Then $X-C$ is an open set. Since by the subspace topology we know that $Y\cap (X-C)$ is open, it implies that $Y\cap X - Y\cap C = Y - C\cap Y = Y-A$ is open in $Y$. Thus $A$ is closed in $Y$. Conversly suppose that if $A$ is closed. Then $Y-A$ is open, and thus can be written as $Y-A = U\cap Y$, by definition of the subspace topology. Since $(X-U)\cap Y = X\cap Y - Y\cap U = Y - A$. Since $U$ is open, $C = X-U$ is closed and thus $A = C\cap Y$.
\end{proof}
\begin{theorem}
  Let $Y$ be a subspace of $X$. If $A$ is closed in $Y$ and $Y$ is closed in $X$ then $A$ is closed in $X$.
\end{theorem}
\begin{proof}
  By the previous theorem we can write $A = C \cap Y$, where $C$ is closed in $X$. Since $Y$ is also closed in $X$ it follows that $A$ is closed (intersection of closed sets is closed).
\end{proof}
\begin{definition}
  The interior of a subset $A$ is of a topological space $X$ is the union of all open sets contained within $A$. This is denoted as $\Int(A)$.
\end{definition}
\begin{definition}
  The closure of $A$ is the intersection of all closed sets containing $A$. This is denoted as $\bar{A}$.
\end{definition}
\begin{lemma}
  If $Y$ is a subspace of topological space $X$, $A$ is a subset of $Y$, and the closure of $A$ in $X$ is $\bar{A}$ then the closure of $A$ in $Y$ is $\bar{A}\cap Y$.
\end{lemma}
\begin{proof}
  Let $B$ denote the closure of $A$ in $Y$. Since $\bar{A}$ is closed in $X$, $\bar{A}\cap Y$ must be closed in $Y$. Since $B$ is the intersection of all closed sets containing $A$ it follows that $B\subset \bar{A}\cap Y$.\\

  Since $B$ is closed in $Y$ it can be written as $C\cap Y$ for some closed set $C$ in $X$. Since $A\subset C$ it follows that $C\subset \bar{A}$ and thus $\bar{A}\cap Y\subset C\cap Y$.
\end{proof}
\begin{theorem}
  Let $A$ be a subset of $X$. Then
  \begin{enumerate}
    \item $x\in \bar{A}$ if and only if every open set $U$ containing $x$ intersects $A$.
    \item If the topology of $X$ is generated by a basis, then $x\in \bar{A}$ if and only if every basis set $B$ containing $x$ intersects $A$.
  \end{enumerate}
\end{theorem}
\begin{proof}
  Consider the first statement. Suppose that $x\notin \bar{A}$. Since $X-\bar{A}$ is open it follows that there exists an open set $U$ containing $x$ that completely contained in $X-\bar{A}$. Thus it does not intersect $A$. Conversly suppose that there exists an open set containing $x$ which does not intersect $A$. Since $X-U$ is a closed set containing $A$, we must have $\bar{A} \in X-U$. Thus $x\notin \bar{A}$. By contrapositive law, we arrive at the statement 1.\\

  Given $1$, if every open set containing $x$ intersects $A$ then so does every basis element since they are open sets. Conversly if every basis element containing $x$ intersects then so does every open set, since it contains a basis containing $x$.
\end{proof}
\begin{notation}
  An open set $U$ containing $x$ will be called a neighborhood of $x$.
\end{notation}
\subsection{Limit Point}
\begin{definition}
  If $A\subset X$ and $x\in X$ then $x$ is a limit point of $A$ if every neighborhood containing $A$ intersects $A$ at a point other than $x$.
\end{definition}
\begin{theorem}
  Let $A'$ be the set of all limit points of $A$. Then $A\cup A' = \bar{A}$.
\end{theorem}
\begin{proof}
  Suppose that $x\in \bar{A}$. If $x\in A$ then clearly $x\in A\cup A'$. If $x\notin A$, then $x$ must be a limit point since every neighborhood of $x$ intersects $A$ at a point other than $x$ (since $x$ is not in $A$). Thus $\bar{A} \subset A\cup A'$. Conversly suppose that $x\in A\cup A'$. Since $A\subset \bar{A}$ if $x\in A$ then $x\in \bar{A}$. If $x\in A'$ then every neighborhood of $x$ intersects $A$ and thus it must be in $\bar{A}$ (by the previous theorem). 
\end{proof}
\begin{theorem}
  A subset is closed if and only if it contains all its limit points. 
\end{theorem}
\begin{proof}
  Suppose $A$ is a closed set. Then $A = \bar{A}$. By previous theorem all limit points of $A$ are contained in $\bar{A}$. Thus they are contained in $A$. Conversly suppose that $A$ contains all its limit points. Since $A = A\cup A' = \bar{A}$ by previous theorem, it follows that $A$ is closed.
\end{proof}
\subsection{Haudorff Spaces}
\begin{definition}
  A sequence $(x_n)\in X$ is said to converge to a point $x\in X$ if every neighborhood $U$ of $x$ there exists $N$ such that $n>N \implies x_n\in U$.
\end{definition}
\begin{example}
  Consider the space $X = \{a,b,c\}$ with topology $ \mathscr{T}= \{\emptyset, \{a,b\}, \{b,c\}, \{b\}, X\}$. Then the sequence $x_n = b$ converges to all $a,b,$ and $c$. This is because every neighborhood of $a$, $b$, and $b$ contains $b$. Thus a sequence may not converge to a single point in a general topological space.
\end{example}
\begin{definition}
  A topological space $X$ is said to be Hausdorff if for each pair of distinct points $x_1,x_2$ there exists neighborhoods $U_1, U_2$ of $x_1,x_2$ respectively such that $U_1\cap U_2 = \emptyset$.
\end{definition}
\begin{theorem}
  Every finite subset of a Hausdorff space is closed.
\end{theorem}
\begin{proof}
  It is enough to prove this for singleton sets $\{x_0\}$ since the finite union of closed sets is closed. Consider any point $x\in X$ distinct from $x_0$. By Hausdorff axiom there exists neighborhood $U$ of $x$ which does not intersect any neighborhood of $x_0$. Thus $x\notin \overline{\{x_0\}}$. Hence $\{x_0\}$ is its own closure.
\end{proof}
\begin{definition}
  A topological space $X$ is said to be $T_1$ if every subset containing finitely many points is closed.
\end{definition}
\begin{theorem}
  Suppose $X$ is a $T_1$ space. Let $A$ be a subset of $X$. Then any neighborhood $U$ of a point $x$ intersects $A$ at infinitely many points if and only if its a limit point.
\end{theorem}
\begin{proof}
  If it intersects at infinitely many points then it clearly intersects at a point other than $x$. Thus $x$ is a limit point.\\

  Suppose that $x$ is a limit point. Let $U$ be a neighborhood of $x$ which intersects $A$ at finitely many points. Let $A\cap U-\{x\} = \{x_1, \cdots, x_n\}$. Since this set is finite it is closed. Thus $X-\{x_1,\cdots, x_n\}$ is open. This means that $U\cap X-\{x_1,\cdots, x_n\}$ is a neighborhood of $x$ which does not intersect $A$. Thus $x$ is not a limit a point, a contradiction.
\end{proof}
\begin{theorem}
  Any sequence in a Hausdorff space converges to atmost one point.
\end{theorem}
\begin{proof}
  Let the sequence $x_n$ converge to $x\in X$. Suppose $y\in X$ is distinct from $x$. Thus there exists a neighborhoods $U_y$ of $y$ and $U_x$ of $x$ which are disjoint. Since there exists $N$ such that $x_n\in U_x$ for all $n>N$, it follows that $x_n \notin U_y$ for all $n>N$. Thus $x_n$ cannot converge to $U_y$.
\end{proof}
