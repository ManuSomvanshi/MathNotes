\section{Rings}
\begin{definition}
  A ring $R$ is a nonempty set along with two binary operations, $+,\cdot : R\times R \to R$ such that
  \begin{enumerate}
    \item $(R,+)$ is an abelian group.
    \item $(R, \cdot)$ is a semigroup; i.e. $\cdot$ is associative.
    \item Multiplication is distributive over addition, i.e. $\forall x,y,z\in R$,
      \begin{align*}
        x(y+z) = xy + xz \And (y+z)x = yx + zx.
      \end{align*}
  \end{enumerate}
\end{definition}
\begin{definition}
  A ring $(R,+, \cdot)$ is a commutative ring if $\cdot$ is commutative. The ring $R$ is said to have an identity $1_R$ if forall $x\in R$, $1_Rx = x1_R = x$. 
\end{definition}
\begin{notation}
  The additive identity of a ring will be denoted by $0$.
\end{notation}
\begin{proposition}
  Let $R$ be a ring. Then
  \begin{enumerate}
    \item $0\cdot x = x\cdot 0 = 0$ for all $x\in R$,
    \item $(-x)\cdot y = x\cdot (-y) = -(x\cdot y)$.
  \end{enumerate}
\end{proposition}
\begin{proof}
  Using the distributivity,
  \begin{align*}
    x\cdot (y +0) &= x\cdot y + x\cdot0\\
    \implies x\cdot y &= x\cdot y + x\cdot 0\\
    \implies x\cdot 0 &= 0.
  \end{align*}
  Similarly $0\cdot x = 0$. Again using distributivity:
  \begin{align*}
    x(y + (-y)) &= xy + x(-y)\\
    \implies xy + x(-y) &= 0\\
    \implies x(-y) &= -(xy).
  \end{align*}
  Similarly $(-x)y = -(xy)$. This completes the proof.
\end{proof}
\begin{definition}
  An element $a$ of a ring $R$ is said to be a \textit{left zero divisor} [resp. right] if $ab = 0$ [resp. $ba=0$] for some $0\neq b\in R$. A zero divisor is one which is both a right and left zero divisor.
\end{definition}
\begin{definition}
  An element $a$ of a ring $R$ with identity is said to be \textit{left invertible} [resp. right] if there exists a $c\in R$ s.t. $ca = 1_R$ [resp. $ac = 1_R$]. The element $c$ is called the left [resp. right] inverse of $a$. If $a$ has both left and right inverse then it said to be a \textit{unit}.
\end{definition}
\begin{remark}
  If $a$ has both a right and left inverse then it both must be equal. This follows from the definition:
  \begin{align*}
    ca = 1_R \And ab = 1_R \implies c(ab) = c \implies (ca)b = c \implies b=c.
  \end{align*}
  Moreover the set of units in a ring $R$ form a group under multiplication.
\end{remark}
\begin{definition}
  A commutative ring $R$ with an identity $1_R \neq 0$ and no zero divisors is called an integral domain. A ring $D$ with an indentity $1_D$ where every non-zero element is a unit is called a division ring. A commutative division ring is called a field.
\end{definition}
\begin{example}
  $\Z$ is a commutative ring with identity and $\Q, \R, \C$ are fields with usual addition and multiplication. The set of $n\times n$ matrices over a field $F$ with matrix addition and multiplication forms a non-commutative ring.
\end{example}
\begin{example}
  $\Z\bign/n\Z$ is a ring for any natural number $n$. When $p$ is a prime $\Z\bign/p\Z$ is a field. For $n\geq 2$ $\Z\bign/n\Z$ is a commutative ring with identity.
\end{example}
\begin{proposition}
  If $x\in \Z_n$ then the following are equivalent:
  \begin{enumerate}
    \item $x$ is a unit.
    \item $x$ has no zero divisor.
    \item $x$ is coprime to $n$.
  \end{enumerate}
\end{proposition}
\begin{proof}
  \textit{$1 \implies 2$.} Suppose that $x$ is a unit. Then there exists $z$ such that $zx\equiv 1 \pmod{n}$. If $xy \equiv 0 \pmod{n}$ then $zxy \equiv 0 \pmod{n}$ and hence $y\equiv 0\pmod{n}$. Similarly it can be shown that $x$ has no left zero divisor.\\
  \textit{$2 \implies 3$.} Assume that $x$ has no zero divisors. Suppose that there exists $d>1$ such that $d|x$ and $d|n$. This means that $x = m_1d$ and $n= m_2d$. Since
  \begin{align*}
    m_2 x &= m_2 m_1 d\\
         &= m_1 m_2 d\\
         &= m_1 n\\
         &\equiv 0\pmod{n}.
  \end{align*}
  Thus there exists $m_2$ such that $m_2 x \equiv 0\pmod{n}$, contrary to the fact that $x$ has no zero divisors. Thus $\gcd(x,n) = 1$.\\
  \textit{$3 \implies 1$.} Supposing that $\gcd(x,n)=1$ it follows by Bezout's identity
  \begin{align*}
    ax+bn =1 \implies ax \equiv 1 \pmod{n}.
  \end{align*}
  Thus $x$ is a unit (using commutative property of $\Z\bign/n\Z$).
\end{proof}
\begin{proposition}
  The following are equivalent:
  \begin{enumerate}
    \item $\Z\bign/n\Z$ is an intergal domain.
    \item $n$ is prime.
    \item $\Z\bign/n\Z$ is a field.
  \end{enumerate}
\end{proposition}
\begin{proof}
  If $\Z\bign/n\Z$ is an integral domain then every $x\in \Z\bign/n\Z$ is a unit. From the previous proposition it follows that $x$ is coprime to $n$ for all $x<n$. This implies that $n$ is a prime.\\ 

  Now suppose that $n$ is prime. It has also been shown that if $\gcd(x,n)=1$ then $x$ is a unit. Since for all $x<n$ $\gcd(x,n) = 1$ it follows that all $x\in \Z\bign/n\Z$ are unit. Since $\Z\bign/n\Z$ is commutative it follows that it's a field.
\end{proof}
\begin{example}
  Let $A$ be an abelian group and let $\text{End}(A)$ denote the endmorphisms of $A$. define $f+g(x) = f(x) + g(x)$ and $f\cdot g(x) = f\circ g(x)$. Clearly $\text{End}(A)$ is a ring with this addition and multiplication. More over its a ring with identity since the identity map is an endomorphism.
\end{example}
\begin{theorem}
  Let $R$ be a ring with identity, $n\in \Z^+$ and $a,b\in R$ then the binomial theorem holds for $a,b$ if $ab=ba$.
\end{theorem}
\begin{proof}
   Consider the $n=1$ case. Clearly this is true since $(a+b)^1 = a+b$. Suppose the statement is true for $n=k$. Then:
   \begin{align*}
     (a+b)^{k+1} &= (a+b)^k (a+b) = \sum_{j=0}^k {}^kC_j a^j b^{k-j} (a+b)\\
              &= \sum_{j=0}^k {}^kC_j a^j b^{k-j} a + \sum_{j=0}^k {}^kC_j a^j b^{k-j+1}\\
              &= \sum_{j=0}^k {}^kC_j a^{j+1} b^{k-j} + \sum_{j=0}^k {}^kC_j a^j b^{k-j+1}\\
              &= a^{k+1} + \sum_{j=1}^k ({}^kC_{j-1} + {}^kC_{j}) a^{j} b^{k-j+1} + b^{k+1}\\
              &= a^{k+1} + \sum_{j=1}^{k} {}^{k+1}C_{j} a^{j} b^{k-j+1} + b^{k+1}\\
              &=\sum_{j=0}^{k+1} {}^{k+1}C_{j} a^{j} b^{k-j+1}\\
   \end{align*}
\end{proof}
\begin{definition}
  A function $f:R\to S$ is a homomorphism of rings if $f$ preserves both addition and multiplication. The kernel of a homomorphism is $\Ker{f} = \{a\in R\ |\ f(a) = 0\}$.
\end{definition}
\begin{example}
  For example the map $\p:\Z\to \Z\bign/n\Z$ given by $\p(a) = a\ \text{mod}n$. This is a ring homomorphism. The kernel of this homomorphism is $\Ker\p = n\Z$.
\end{example}
\begin{example}
  Consider the map $\p:\Z\bign/3\Z \to \Z\bign/6\Z$ given by $x\mapsto 4x\text{mod}6$. If $x,y\in \Z\bign/3\Z$ then
  \begin{align*}
    \p(xy) \equiv 4xy\text{mod}6 \equiv 16 xy \text{mod}6 = \p(x)\p(y),
  \end{align*}
  and
  \begin{align*}
    \p(x+y) \equiv 4(x+y)\text{mod}6 = \p(x) + \p(y) 
  \end{align*}
  Thus $\p$ is a ring homomorphism. Notice that $\p(1) = 4$. This shows that ring homomorphisms need not preserve identities.
\end{example}
\begin{definition}
  Let $R$ be a ring and $n$ be the least positive integer such that $na = 0$ for all $a\in R$ then $R$ is called a ring of characteristic $n$. If no such $n$ exists then $R$ is said to have characteristic $0$. (Notation: $\char R = 0$).
\end{definition}
\begin{theorem}
  Let $R$ be a ring with identity $1_R$ and characteristic $n>0$. 
  \begin{enumerate}
    \item $n$ is the least positive integer such that $n1_R = 0$.
    \item If $\p:\Z \to R$ is a map given by $m\mapsto m1_R$ then $\p$ is a homomorphism with kernel $\Ker \p = \{kn\in \Z\ |\ k\in \Z\}$.
    \item $n$ is the smallest positive integer such that $n1_R = 0$.
    \item If $R$ is an integral domain then $n$ is prime.
  \end{enumerate}
\end{theorem}
\begin{proof}
  Let $S = \{k\in \Z^+\ |\ k1_R = 0\}$. $S$ is nonempty since $n\in S$. Clearly $S$ has a least element, $m$, by well ordering principle. For any $x\in R$
  \begin{align*}
    ma = (m1_R) a = 0.
  \end{align*}
  Thus $R$ is of characteristic $m$, which is a contradiction unless $n=m$. Moreover for any $k\in S$, let $k = qn+r$ where $0\leq r<n$. Since $k1_R = 0$ it follows that $r 1_R = 0$. Again this is a contradiction unless $r = 0$. Thus $n|k$ for all $k\in S$.\\
  Let $k,m\in \Z$, then:
  \begin{align*}
    \p(km) = km 1_R = k\p(m) = k1_R \p(m) = \p(k)\p(m)
  \end{align*}
  and
  \begin{align*}
    \p(k+m) = (k+m)1_R = k1_R + m1_R = \p(k) + \p(m). 
  \end{align*}
  Thus $\p$ is a ring homomorphism. The kernel would be
  \begin{align*}
    \Ker \p &= \{a\in \Z\ |\ \p(a) = 0\}\\
           &= \{a\in \Z\ |\ a1_R = 0\} \\
           &= \{a\in \Z\ |\ a = qn,\ q\in \Z\}\\
  \end{align*}
  This proves $2$.\\
  Suppose that $R$ is an integral ring and that $n = qd$ for some $d,q$. Then $dq 1_R = 0$ which further implies that $d1_R = 0$ or $q1_R = 0$ (since there are no zero divisors). This is again a contradiction since $n$ is the smallest number such that $n1_R = 0$.
\end{proof}
\begin{theorem}
  Every ring $R$ may be embedded (a ring monomorphism) in a ring $S$ with identity. The ring $S$ may be chosen to have characteristic $0$ or $\char R$.
\end{theorem}
\begin{proof}
  Let $S = R\oplus \Z$ and let the addition and product be defined as:
  \begin{align*}
    (r_1, k_1) + (r_2, k_2) = (r_1+r_2, k_1+k_2)\\
    (r_1,k_1)(r_2,k_2) = (r_1r_2 +k_1r_2 + k_2r_1, k_1 k_2) 
  \end{align*}
  It can be easily checked that this product well defined and satisfies the necassary properties. Consider the cannonical map $\p:R\to R\oplus \Z$ given by $r\mapsto (r,0)$. Let $r_1, r_2\in R$ then
  \begin{align*}
    \p(r_1r_2) = (r_1r_2,0) = (r_1, 0)(r_2,0) = \p(r_1)\p(r_2)\\
    \p(r_1 + r_2) = (r_1+r_2,0) = (r_1,0) + (r_2,0) = \p(r_1)+\p(r_2).
  \end{align*}
  Thus $\p$ is a ring homomorphism. Since the map is one-one it follows that $\p$ is an embedding. Suppose $R$ has characteristic $n>0$. Then the $S$ constructed above has characteristic $0$, since $\Z$ has characteristic $0$. To get a characteristic $n$ ring, let $S = R\oplus \Z\bign/n\Z$.
\end{proof}
