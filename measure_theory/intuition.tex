\section{Intuition of Measure}
Consider any interval of $\R$, $(a,b]$. Intuitively we define the length of this interval as $\lambda\l((a,b]\r) = b-a$. Now is it possible to extend this concept of length to any subset of $\R$? For that we would wish to find a function $\lambda$ such that the following properties are true:
\begin{enumerate}
  \item $\lambda: \mathfrak{P} (\R)\to \R^{+}$ is a set function.
  \item If $I = (a,b]$ is any interval in $\R$ then $\lambda(I) = b-a$.
  \item Length of union of two disjoint subsets $A, B\in \mathfrak{P}(\R)$ must be the sum of their individual lengths, i.e. $\lambda(A+B) = \lambda(A)+\lambda(B)$. This can be extended to any countable union of pairwise disjoint subsets. 
  \item The translation of any subset $A\in \mathfrak{P}(\R)$ must have the same length, i.e. $\forall\ x\in \R,\ \lambda(A+x) = \lambda(A)$. 
\end{enumerate}
\begin{conjecture}
  The function $\lambda$ as described above does not exist.
  \label{con:not-exist}
\end{conjecture}
To prove this we must first prove some other propositions. We use the following notations: let $\sim$ be an equivalence relation on $\R$ defined as:
  \[x\sim y\ \text{if}\ x-y\in\mathbb{Q}\]
  Let $[x]$ denote the equivalence classes of $x$, and let $\Lambda = \R/\sim$. 
\begin{proposition}
 The set $\Lambda$ is uncouontable.  
 \label{uncountable}
\end{proposition}
\begin{proof}
  Let $\alpha\in\Lambda$ be an equivalence class. Let $x\in\alpha$ be a fixed point. Then for each $y\in\alpha$ it is possible to find a unique rational number given by $x-y$. Hence $\alpha$ is a countable set. Since the countable union of countable sets is countable, but $\R$ is uncountable, it follows that $\Lambda$ must be uncountable. 
\end{proof}
Let $\Omega\subset \R$ be set constructed in the following way: for each $\alpha\in\Lambda$ we know that a point can be found between $(0,1)$; so take one such point from each $\alpha$ and put it in the set $\Omega$. From this construction it is easy to see that $\Omega\subset (0,1)$.
\begin{proposition}
  Let $p,q\in \mathbb{Q}$, then either $\Omega+q=\Omega+p$ or $\Omega+q\cap\Omega+p=\emptyset$.
  \label{trans_eq}
\end{proposition}
\begin{proof}
  Let's say $\Omega+q\cap\Omega+p\neq\emptyset$. Then for $a,b \in \Omega$ we can find an $x\in\Omega+q\cap\Omega+p$ such that $x=a+p=b+q$. Hence $a-b = q-p$ for all $a,b\in\Omega$. Since $q-p$ is rational, $a-b$ will also be rational. Hence $a$ and $b$ belong to the same equivalence class. But since we only chose one element from each equivalence class in the construction of $\Omega$, we must have $a=b$. Hence $q=p$, making the two sets in question equal. 
\end{proof}
Hence from this proposition we can say that if $q\neq p$ then $\Omega+q\cap\Omega+p=\emptyset$.
\begin{proposition}
  Let $\lambda$ be a length function as defined above. If $A\subset B\subset\R$ then $\lambda(A)\leq \lambda(B)$.
  \label{measure prop}
\end{proposition}
\begin{proof}
 Since $B = A \cup (B-A)$, and $A$ and $B-A$ are disjoint,
 \begin{align*}
   \lambda(B) &= \lambda(A \cup (B-A))\\
        &= \lambda(A) + \lambda(B-A)\\
        &\geq \lambda(A)
 \end{align*}
 Hence proving our claim.
\end{proof}
Now we are ready to prove \cref{con:not-exist}.
\begin{proof}[proof of \cref{con:not-exist}]
  Consider the union of sets $\Omega+q$:
  \[\underset{-1<q<1}{\bigcup_{q\in \mathbb{Q}}}\Omega+q\]
  From proposition \ref{trans_eq} we know that this is a union of disjoint sets. Hence,
  \[\lambda\l(\underset{-1<q<1}{\bigcup_{q\in \mathbb{Q}}}\Omega+q\r) = \underset{-1<q<1}{\sum_{q\in \mathbb{Q}}}\lambda\l(\Omega+q\r)\]
  Using property $4$ of $\lambda$, 
  \[\lambda\l(\underset{-1<q<1}{\bigcup_{q\in \mathbb{Q}}}\Omega+q\r) = \underset{-1<q<1}{\sum_{q\in \mathbb{Q}}}\lambda\l(\Omega\r) = 0\]
  Let $x\in (0,1)$. Let $a\in \Omega\cap [x]$. Then we know that $x-a = q$ for some rational $q$. Since $a\in\Omega$ implies $a\in (0,1)$, the range of $q$ must be $-1<q<1$. Hence $x = a + q$ for some $-1<q<1$ implying that
 \[x \in \underset{-1<q<1}{\bigcup_{q\in \mathbb{Q}}}\Omega+q\]
 further implying that,
 \[(0,1) \subset \underset{-1<q<1}{\bigcup_{q\in \mathbb{Q}}}\Omega+q\]
 Using proposition \ref{measure prop},
 \[1 \leq \lambda\l(\underset{-1<q<1}{\bigcup_{q\in \mathbb{Q}}}\Omega+q\r)\]
 Hence we have arrived at a contradiction. This shows that a function $\lambda$ with the properties 1,2,3,4 as given above does not exist.
\end{proof}
Hence this shows that to construct a general notion of length (called the \textit{measure}) we must let go of one of the four properties: 1,2,3, or 4. Since $2,3,4$ are essential for a notion of length, we change 1 to be the following:
\begin{enumerate}
  \item  $\lambda: \mathscr{B}(\subset\mathfrak{P} (\R))\to \R^{+}$ is a set function.
\end{enumerate}
This means that we are discarding the notion that all subsets of $\R$ can be assigned a length. 
