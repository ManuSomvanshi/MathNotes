\section{Metric Spaces and Euclidean Space}
\begin{definition}
  The pair $(X,d)$ is said to be a \textit{metric space} where $X$ is some non-empty set and $d:X\times X\to \R_{\geq 0}$ is a function with the following properties:
\begin{enumerate}
  \item $d(x,y) = 0 \iff x = y$.
  \item $d(x,y) = d(y,x)\ \forall x,y\in X$.
  \item $d(x,y) \leq d(x,z) + d(z,y)$.
\end{enumerate}
\end{definition}
\begin{definition}
  Let $(X,d)$ be a metric space. Then:
  \begin{enumerate}
    \item A \textit{neighborhood} of a point $x\in X$ is the set $N_r(x) \equiv \{p\ |\ d(x,p) <r\}$.
    \item A point $p$ is a \textit{limit point} of set $E\subset X$ if \textit{every} neighborhood of $p$ contains a $q\neq p$ s.t. $q\in E$.
    \item If $p\in E$ and $p$ is not a limit point of $E$ then $p$ is called an \textit{isolated point} of $E$.
    \item $E$ is closed if every limit point of $E$ is in $E$.
    \item $p$ is said to be in the interior of $E$ if $\exists\ \e>0$ s.t. $N_\e(p)\subset E$.
    \item $E$ is open if every point of $E$ is an interior point of $E$.
    \item $E$ is perfect if $E$ is closed and every point of $E$ is a limit point of $E$.
    \item $E$ is bounded if there exists an $M\in \R$ and $q\in X$ such that $d(p,q) < M,\ \forall\ p\in E$.
    \item $E$ is dense in $X$ if every point of $X$ is either a limit point of $E$, a point in $E$, or both.
  \end{enumerate}
\end{definition}
