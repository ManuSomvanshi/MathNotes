\section{More on Permutation Group}
\begin{definition}
  A $2$-cycle is an element of $S_n$ which can be written as $(a_1a_2)$. A $2$-cycle is called a transposition.
\end{definition}
\begin{definition}
  Two cycles $(a_1...a_k), (b_1...b_\ell)\in S_n$ are said to be disjoint if $a_i \neq b_j$ for all $i,j$.
\end{definition}
\begin{proposition}
  Disjoint cycles commute under composition.
\end{proposition}
\begin{proof}
  Let $f \doteq (a_1...a_k), g\doteq (b_1...b_\ell)$ be disjoint cycles then consider the composition $g\circ f$. Due to the disjointness any number $m$ is either moved by $f$ alone, by $g$ alone, or not moved at all. If $m$ is not moved by both then clearly $f\circ g(m) = g\circ f(m)$. If it is moved by $f$ only, then $\exists\ j$ such that $f(m) = a_j$. It further follows that $f\circ g(m) = a_j = g(a_j) = g\circ f(m)$. Similarly this can be shown if $m$ is only moved by $g$. Therefore $f\circ g = g\circ f$.
\end{proof}
\begin{proposition}
  Set of all transpositions generates the group $S_n$.
\end{proposition}
\begin{proof}
  Any element of $S_n$ can be written as product of cycles. All that remains to be shown is that any cycle can be written as composition of transpositions. Since,
  \[(a_1...a_k) = (a_1a_k)...(a_1a_3)(a_1a_2).\]
  This completes the proof.
\end{proof}
\begin{proposition}
  The set $\{(1k)\ |\ 2\leq k\leq n\}$ generates $S_n$.
\end{proposition}
\begin{proof}
  Since any element can be written as a composition of transpositions, all that we need to show is that any transposition can be written in terms of $(1k)$. Since,
  \[(ab) = (1a)(1b)(1a).\]
  This proof is complete.
\end{proof}
\begin{proposition}
  The set $\{(kk+1)\ |\ 1\leq k\leq n-1\}$ generates $S_n$.
\end{proposition}
\begin{proof}
  Using the above proposition all we need to show is that $(1a)$ can be written in terms of $(kk+1)$. Since,
  \[(1a) = (a-1a)...(23)(12)(23)...(a-1a)\].
  This proof is complete.
\end{proof}
\begin{proposition}
  The set $\{(12), (12...n)\}$ generates $S_n$.
\end{proposition}
\begin{proof}
  This will be proven by showing that $(aa+1)$ can be written in terms of $(12)$, $(1...n)$ and their powers. Since the map $(1...n)^{a-1}$ takes $1\to a$ and the map $(1...n)^{1-a}$ takes $a$ to $1$. Thus,
  \[(aa+1) = (1...n)^{a-1}(12)(1...n)^{1-a}.\]
  The compositions works as follows: $a\to 1 \to 2 \to a+1$. This completes the proof.
\end{proof}
\begin{definition}
  An element of $S_n$ is said to be \textit{even} if it can be written as a product of even number of transpositions. Similarly element is said to be \textit{odd}. 
\end{definition}
\begin{proposition}
  Any element of $S_n$ is either even or odd.
\end{proposition}
\begin{proof}
 Define the polynomial $P$ as:
 \[P(x_1,...,x_n) = \prod_{i=1}^n \prod_{j>i}^n (x_i - x_j)\]
If $\alpha\in S_n$ then define $\alpha P$ as:
\[\alpha P(x_1,...,x_n) = \prod_{i=1}^n \prod_{j>i}^n (x_{\alpha(i)} - x_{\alpha(j)})\]
The terms in the polynomial $\alpha P$ are the same as $P$, the only difference would be the order of some may change, introducing a sign. Thus $\alpha P = P$ or $\alpha P = -P$. Clearly if $\alpha,\beta\in S_n$ then the sign change introduced would be the product of the sign change introduced by each. The sign introduced by the transposition $(ab)$ is $-1$ (the only term that changes sign will be $x_a-x_b$), thus if $\alpha\in S_n$ is odd then $\alpha$ changes $P$ by $-1$, on the other hand if $\alpha$ is even then it does not change the sign of $P$. Since $\alpha$ is independent of the way we chose to represent it as products of transpositions, $\alpha P = \pm P$ will also be independent of the representation. Thus if $\alpha$ is even or odd in one representation it must be in all.
\end{proof}
\begin{definition}
  The set of all even elements of $S_n$ forms a subgroup of order $n!/2$ called the alternating group $A_n$.
\end{definition}
\begin{proposition}
  $A_n$ is generated by $3-$cycles.
\end{proposition}
\begin{proof}
  Every three cycle $(abc)$ can be expressed as $(ac)(ab)$, and thus is even. Any element of $A_n$ can be expressed in terms of products of even number of $(1a)$. Pair the adjacent transpositions in the following way: $(1a)(1b) = (1ba)$. Thus every element can be written as products of $3-$cycles. 
\end{proof}
