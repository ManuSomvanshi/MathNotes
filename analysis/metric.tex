\section{Metric Spaces and Euclidean Space}
\begin{definition}
  The pair $(X,d)$ is said to be a \textit{metric space} where $X$ is some non-empty set and $d:X\times X\to \R_{\geq 0}$ is a function with the following properties:
\begin{enumerate}
  \item $d(x,y) = 0 \iff x = y$.
  \item $d(x,y) = d(y,x)\ \forall x,y\in X$.
  \item $d(x,y) \leq d(x,z) + d(z,y)$.
\end{enumerate}
\end{definition}
\begin{definition}
  Let $(X,d)$ be a metric space. Then:
  \begin{enumerate}
    \item A \textit{neighborhood} of a point $x\in X$ is the set $N_r(x) \equiv \{p\ |\ d(x,p) <r\}$.
    \item A point $p$ is a \textit{limit point} of set $E\subset X$ if \textit{every} neighborhood of $p$ contains a $q\neq p$ s.t. $q\in E$.
    \item If $p\in E$ and $p$ is not a limit point of $E$ then $p$ is called an \textit{isolated point} of $E$.
    \item $E$ is closed if every limit point of $E$ is in $E$.
    \item $p$ is said to be in the interior of $E$ if $\exists\ \e>0$ s.t. $N_\e(p)\subset E$.
    \item $E$ is open if every point of $E$ is an interior point of $E$.
    \item $E$ is perfect if $E$ is closed and every point of $E$ is a limit point of $E$.
    \item $E$ is bounded if there exists an $M\in \R$ and $q\in X$ such that $d(p,q) < M,\ \forall\ p\in E$.
    \item $E$ is dense in $X$ if every point of $X$ is either a limit point of $E$, a point in $E$, or both.
  \end{enumerate}
\end{definition}
\begin{proposition}
  A point $x$ is a limit point of $E\subset X$ if and only if there exists a sequence $(x_n)\in E$ which converges to $x$.
\end{proposition}
\begin{proof}
  Suppose $x$ is a limit point. Then every neighborhood $N_{1/n}(x)$ $\exists\ x_n (\neq x) \in E$ such that $x_n \in N_{1/n}(x)$. Since,
  \begin{align*}
    d(x_n, x) < 1/n \implies \lim_{n\to \infty} x_n = x.
  \end{align*}
  Suppose that there exists a sequence $(x_n) \in E$ such that $x_n \to x$. Let $N_\e(x)$ be some neighborhood of $x$. Since there exists $N$ such that $n>N \implies d(x,x_n)<\e$ it follows that for all $n>N$, $x_n \in N_\e(x)$.
\end{proof}
\begin{proposition}
  Every neighborhood is open.
\end{proposition}
\begin{proof}
  Let $x\in X$ be some point and let $N_r(x)$ be some neighborhood of $x$. Let $p\in N_r(x)$. Choose $\e < r- d(x,p)$. Then for any $y\in N_\e(p)$, 
  \[d(x,y) \leq d(x,p) + d(p,y) < r.\]
  Hence every point $p$ is in the interior.
\end{proof}
\begin{proposition}
  If $p$ is a limit point of $E$ then there are infinitely many points of $E$ in any neighborhood of $p$.
\end{proposition}
\begin{proof}
  Assume that some neighborhood has finitely many points of $E$, given by the set $S= \{y_1,...,y_n\}$. Then let $\delta < \min\{d(p,y_i)\ |\ y_i \in S\}$. Then there exists another point $y\in N_\delta (p)$ such that $y\in E$ (since $p$ is a limit point). This is a contradiction, hence there are infinite points in every neighborhood.
\end{proof}
\begin{corollary}
  A finite set of points has no limit points.
\end{corollary}
\begin{proof}
  If it had a limit point, then every neighborhood of that point must have infinite points. This isnt possible since it has only finitely many points.
\end{proof}
\begin{proposition}
  Let $\{E_\a\}$ be any collection of points then:
  \[\bigcap_\a E_\a^c = \l(\bigcup_\a E_\a\r)^c.\]
\end{proposition}
\begin{proof}
  It's simple. Just show if $x\in A$ then $x\in B$ and the converse, where $A,B$ are the LHS and RHS of the above equation respectively. 
\end{proof}
\begin{proposition}
  A set is open iff it's compliment is closed.
\end{proposition}
\begin{proof}
  Suppose $E^c$ is closed. Let $x\in E$ then $x\notin E^c$ which means that $x$ is not a limit point of $E^c$. Thus there exists a neighborhood $N$ of $x$ such that $N\cap E^c = \emptyset$. Thus $N\subset E$. Thus $x$ has a neighborhood contained in $E$, making it an interior point. Thus $E$ is open.\\

  Suppose $E$ is open. Let $x$ be a limit point of $E^c$. Then for any neighborhood $N$ of $x$ there exists a point $p$ s.t. $p\in N\cap E^c$. Thus no neighborhood of $x$ is contained in $E$, thus $x$ is not an interior point of $E$ and since by assumption $E$ is open it follows that $x\in E^c$. Thus $E^c$ is closed.
\end{proof}
\begin{corollary}
  A set is closed iff it's compliment is open.
\end{corollary}
\begin{proof}
  Directly follows from previous proposition.
\end{proof}
\begin{proposition}
  Let $X$ be a metric space and $\{G_\a\}$ be any collection of subsets, then:
  \begin{enumerate}
    \item If $\{G_\a\}$ is open then $\bigcup_\a G_\a$ is open.
    \item If $\{G_\a\}$ is closed then $\bigcap_\a G_\a$ is closed.
    \item If the collection $\{G_\a\}$ is finite and each set is open then $\bigcap_{\a=1}^n G_\a$ is open.
    \item If the collection $\{G_\a\}$ is finite and each set is closed then $\bigcup_{\a=1}^n G_\a$ is closed.
  \end{enumerate}
\end{proposition}
\begin{proof}
  \begin{enumerate}
    \item Let $x\in \bigcup_\a G_\a$ then there exists $\alpha$ s.t. $x\in G_\a$. Since $G_\a$ is open, every neighborhood $N$ of $x$ is contained in $G_\a$ and therefore in $\bigcup_\a G_\a$. Proving that $\bigcup_\a G_\a$ is open.
    \item Let $x\in X$ be a limit point of $\bigcap_\a G_\a$. Then for every neighborhood $N$ of $x$, $N\cap \bigcap_\a G_\a$ has infinitely many points. Thus $N\cap G_\a$ has infinitely many points (forall $\a$). This means that $x$ is a limit point of every $G_\a$. Since $G_\a$ is closed $x\in G_a$, for all $\a$. Therefore $x\in \bigcap_\a G_\a$.
    \item Let $x\in \bigcap_{\a=1}^n G_\a$ then for each $\a$ there exists a neighborhood $N_\a$ of $x$ such that $N_\a\subset G_\a$. Let $r$ be the minimum of the radii of the neighborhoods $N_\a$, then $N_r(x) \subset \bigcap_{\a=1}^n G_\a$. Thus every $x$ is in the interior of $\bigcap_{\a =1}^n G_\a$.
    \item Just use the above proof for compliments, and then use de-morgan law.
  \end{enumerate}
\end{proof}
\begin{definition}
  Let $(X,d)$ be a metric space, $E\subset X$, $L$ be the set of limit points of $E$ then $\bar{E} \coloneq E\cup L$ is called the closure of $E$.
\end{definition}
\begin{proposition}
  The closure of any set is closed.
\end{proposition}
\begin{proof}
  It is obvious since every limit point is in the set by definition.
\end{proof}
\begin{proposition}
  $E=\bar{E}$ iff $E$ is closed.
\end{proposition}
\begin{proof}
  If $E$ is closed then $L\subset E \implies \bar{E} = E\cup L = E$. If $\bar{E} = E$ then $\bar{E}\cap L = E\cap L$. Since $L\subset \bar{E}$ it follows that $L = E\cap L$. Thus $L\subset E$.
\end{proof}
\begin{proposition}
  If $E\subset F$ and $F$ is closed then $\bar{E} \subset F$.
\end{proposition}
\begin{proof}
  Let $x$ be a limit point of $E$. Since $E\subset F$, it follows that $x$ is also a limit point of $F$ (since it every neighborhood of $x$ would contain a point $E$ which is also in $Y$). Since $F$ is closed every limit point of $F$ is in $F$. Thus if $L$ is the set of limit points of $E$ then $L\subset F$. Hence $\bar{E} = E\cup L \subset F$.
\end{proof}
\begin{proposition}
  If $E\subset Y\subset X$ then $E$ is open relative to $Y$ iff $E = G\cap Y$ for some open set in $G$ in $X$.
\end{proposition}
\begin{proof}
  Suppose that $E = G\cap X$. Let $x\in E$. Then there exists a neighborhood $N_r(x)$ such that $N_r(x)\subset G$. Consider $N = N_r(x)\cap G$. Clearly this is a neighborhood of $x$ relative to $Y$. Since $N\subset E$, $x$ is an interior point of $E$.\\

  Suppose $E$ is open relative to $Y$. Then $\exists\ r_x>0$ s.t. $V_{r_x}(x) \coloneqq \{p\ |\ d(x,p)<r_x,\ p \in Y\} \subset E$. Clearly $V_{r_x}(x) = N_{r_x}(x) \cap Y\subset E$. Let $G = \bigcup_{x\in E} N_{r_x}(x)$. Clearly $G$ is open. Thus $G\cap Y = E$ since $\bigcup_{x\in E} V_{r_x}(x) = E$.
\end{proof}
\begin{definition}
  A collection $\{G_\a\}$ is said to be an open coering of $E\subset X$ if $G_\a$ are covered and $E\subset \bigcup_{\a} G_\a$. A subcovering is a subset of $\{G_\a\}$ which also covers $E$. 
\end{definition}
\begin{definition}
  A subset $Y$ of metric space $X$ is said to be compact if every open covering of $Y$ contains a finite subcover.
\end{definition}
\begin{proposition}
  If $K\subset Y\subset X$ then $K$ is compact in $X$ iff $K$ is compact in $Y$.
\end{proposition}
\begin{proof}
  Suppose $K$ is compact in $X$. Let $\{H_\a\}$ be any open covering of $K$ in $Y$. Then by the previous theorem each set $H_\a = G_\a \cap Y$, where $G_\a$ is open in $X$. Since $K$ is compact in $X$ there exists a finite subcover $\{G_1,...G_n\}$. The collection $\{H_1=G_1\cap Y, ..., H_n G_n\cap Y\}$ will be a finite subcover of $K$ in $Y$. Thus there is a finite subcover of every open cover in $Y$.\\

  Conversly, suppose that $K$ is compact relative to $Y$, then similarly using the same theorem it is possible to construct a finite subcovering of any opern cover of $K$ in $X$.
\end{proof}
\begin{proposition}
  Compact subsets of $X$ are closed. 
\end{proposition}
\begin{proof}
  We will prove that the compliment of a compact subset $K$ is open. Let $p\in K^c$ and $q\in K$. Let $\e_q < d(p,q)/2$. Then the union $\bigcup_{q\in K} N_{\e_q}(q)$ is an open covering of $K$. Since $K$ is compact there some finite $q_1,...,q_n$ s.t. $\{N_{\e_i}(q_i)\}$ is also a covering of $K$ (I have defined $\e_i = d(p,q_i)$). Let $G = \bigcap_{i = 1}^n N_{\e_i}(p)$. $G$ is an open neighborhood of $p$. Let $x\in G$ then $d(x,p)<d(p,q_i),\ \forall\ 1\leq i\leq n$. Since:
  \begin{align*}
    d(p,q_i) &\leq d(p,x) + d(q_i,x)\\
    d(p,q_i) &< \f{1}{2}d(p,q_i) + d(q_i,x)\\
    \implies \f{1}{2}d(p,q_i) < d(q_i,x).
  \end{align*}
  Thus $x\notin N_{\e_i}(q_i)$, and thus $x\notin K$. Meaning that $G\cap K = \emptyset$. Since $G$ is an open neighborhood of $p$, and $G\subset K^c$ it follows that $K^c$ is open.
\end{proof}
\begin{proposition}
  Closed subsets of compact sets are compact.
\end{proposition}
\begin{proof}
  Let $H\subset K$, where $K$ is compact and $H$ is closed. Let $\Omega$ be an open cover of $H$. Then $\Omega \cup \{H^c\}$ is also an open covering of $K$ (this only works since $H^c$ is open). Since $K$ is compact there exists a finite subcovering $\Phi$. If $H^c \in \Phi$ then $\Phi - \{H^c\}$ is a finite subcovering of $H$.
\end{proof}
\begin{corollary}
  If $F$ is closed and $K$ is compact then $F\cap K$ is compact.
\end{corollary}
\begin{proof}
  Since $K,H$ are closed $K\cap H$ is closed, and a subset of $K$. Thus it must be compact.
\end{proof}
\begin{proposition}
  Let $ \mathscr{K} = \{K_\a\}$ be a collection of compact sets such that every subcollection has non-empty intersection. Then $\bigcap_{\a} K_\a \neq \emptyset$.
\end{proposition}
\begin{proof}
  Suppose that the intersection is empty. Then there exists $K_1\in \mathscr{K}$ s.t. $K\cap K_\a = \emptyset$ whenever $K_\a\neq K_1$. This means that $K_1 \subset K^c_\a$. Thus $\{K^c_\a\}$ form an open cover of $K_1$. This implies that there exists a finite subcover $\{K^c_{\a_i}\ |\ 0 \leq i\leq n\}$ which covers $K_1$. Thus $K_1 \cap K_{\a_1} \cap ... \cap K_{\a_n} = \emptyset$, which is a contradiction to the hypothesis.
\end{proof}
\begin{corollary}
  If $\{K_n\}$ are compact sets and $K_n \supset K_{n+1}$ then $\bigcap_{n\geq 1} K_n$ is non-empty.
\end{corollary}
\begin{proposition}
  Let $E\subset K$ where $K$ is compact and $E$ is infinte, then $E$ has atleast one limit point in $K$.
\end{proposition}
\begin{proof}
  Suppose there exists no limit point of $E$ in $K$. This means that $\forall\ q\in K\ \exists\ N(q)$ s.t. $N(q)$ has at most one point of $E$ (i.e. $q$). It is clear that $\{N(q)\}$ forms an open covering of $K$. Since $E$ is infinite and $N(q)$ only has upto one point of $E$ it is not possible to find a finite subcovering of $E$. Thus $\{N(q)\}$ has no finite subcovering of $K$ (since $E\subset K$). This is contrary to the fact that $K$ is compact.
\end{proof}
\begin{proposition}
  If $I_n$ are non-empty intervals in $\R$ and $I_n \supset I_{n+1}$ then $\bigcap_{n\geq 1} I_n$ is non-empty.
\end{proposition}
\begin{proof}
  Let $I_n = [a_n,b_n]$. Consider the set $\{a_n\}$. Clearly this in non-empty and bounded above by $b_1$. Let the supremum of the set be $x$. Since
  \[a_n < b_m\]
  for any $m$, it follows that $x\leq b_m$. We also know that $a_m \leq x$. Thus $x\in I_m$ for all $m$.
\end{proof}
\begin{corollary}
  Let $I_n$ be a sequence of non-empty $k-$cells such that $I_n \supset I_{n+1}$.
\end{corollary}
\begin{proof}
  Follows directly from the proposition above.
\end{proof}
\begin{theorem}
  Every $k-$cell is compact.
\end{theorem}
\begin{proof}
  Let $I$ be a $k-$cell with points $(x_1,...,x_k)$ such that $a_j \leq x_j \leq b_j$. Define $\e$ as:
  \[\e = \l(\sum_{j=1}^n (b_j-a_j)^2\r)^{1/2}.\]
  Clearly if $x,y\in I$ then $||x-y||<\e$. Suppose that $k-$cells are not compact. Then there exists an open cover $\Omega$ which does not have a finite subcover of $I$. Let $c_j = (a_j + b_j)/2$, then cartesian products of $[a_j, c_j]$ and $[c_j,b_j]$ in different combinations produce $2^k$ $k-$cells. At least one of these $k-$cells cannot be covered using a finite subcovering of $\Omega$, call this $I_1$. Repeat the same process for $I_1$ to gain $I_2$ and so on. Thus we have a sequence of $k-$cells such that:
  \begin{enumerate}
    \item $I\supset I_1 \supset I_2 \cdots$;
    \item Each $I_n$ does not have a finite subcovering in $\Omega$;
    \item If $x,y\in I_n$ then $||x-y|| < 2^{-n}\e$.
  \end{enumerate}
  By the above proposition we know that there exists at least one $x^*$ s.t. $x^*\in I_n$ for all $n$. Since $\Omega$ covers $I$ there exists $G\in \Omega$ s.t. $x^*\in G$. Since $G$ is open there exists $r>0$ s.t. $N_r(x^*)\subset G$. Choose $n$ large enough so that $2^{-n}\e < r$. This means that $I_n \subset N_{2^{-n}\e}(x^*) \subset G$. This is a contradiction since $G$ alone covers $I_n$.
\end{proof}
\begin{theorem}[Heine-Borel Theorem]
  If $E$ is a subset of $\R^k$ then the following are equivalent:
  \begin{enumerate}
    \item $E$ is bounded and closed.
    \item $E$ is compact.
    \item Every infinite subset of $E$ has a limit point in $E$.
  \end{enumerate}
\end{theorem}
\begin{proof}
  $2$ follows from $1$ since every bounded set can be contained inside a $k$-cell, and the fact that every closed subset of a compact set is compact. $3$ follows from $2$ due to proposition $2.23$. All that is remaining is to show $1$ from $3$.\\
  If $E$ is not bounded then there is a sequence in $x_1,...,x_n\in E$ such that $|x_n| > n$ for each $n$. Clearly this does not have a limit point in $\R^k$, hence does not have a limit point in $E$. Thus 3 implies that $E$ must be bounded. If $E$ is not closed then there is a point $x_0$ which is a limit point of $E$ but not in $E$. This means that there is a sequence in $(x_n)\in E$ which converges to $x_0$. Since every infinite subset of $E$ has a limit point in $E$, we have a contradiction. Thus $E$ must be closed.
\end{proof}
\begin{theorem}
  Every bounded infinite subset of $\R^k$ has a limit point in $\R^k$.
\end{theorem}
\begin{proof}
  If $E$ is a bounded subset of $\R^k$ then it is contained inside a $k$-cell $I$. Since $I$ is compact every limit point of every subset is contained in $I$ and thus in $\R^k$.
\end{proof}
\begin{definition}[Seperable Sets]
  Two subsets $A,B \subset X$ are said to be seperated if $\bar{A}\cap B = A\cap \bar{B} = \emptyset$.
\end{definition}
\begin{definition}[Connected]
  A subset $A\subset X$ is said to be connect if \textit{cannot} be written as the disjoint union of non-empty seperated sets.
\end{definition}
\begin{theorem}
  $E\subset \R$ is connected if and only if $x,y \in E \And x<z<y \implies z\in E$.
\end{theorem}
\begin{proof}
  Suppose that $E$ is connected and there exist $x,y\in E$ such that there is a $z\in (x,y)$ and $z\notin E$. Then let $A= (-\infty, z)\cap E$ and $B= (z,\infty) \cap E$. Since $A\subset (-\infty,z)$ and $B\subset (z,\infty)$ it follows that $A$ and $B$ are seperated. Since $A\cup B = E$, we have arrived at a contradiction.\\

  To prove the converse suppose that $E$ is not connected. Then $E = A\cup B$ where $A$ and $B$ are seperated. Let $x\in A$ and $y\in B$ and w.l.o.g. assume $x<y$. Let $z =\sup(A\cap [x,y])$. This means that $z$ is a limit point of $A$ and thus $z\in \bar{A}$. Since $A$ and $B$ are seperated $z\notin B$. Hence $x\leq z < y$.
  \begin{enumerate}
    \item If $z\notin A$ then $x<z<y$ and $z \notin E$.
    \item If $z\in A$; then $z\notin \bar{B}$. Since $z$ is not a limit point of $B$ there is an open neighborhood $(z,z_1) \cap B = \emptyset$. Thus $z_1 \notin B$. Since $z$ is a supremum of $A\cap [x,y]$ and $z_1\in [x,y]$, it follows that $z_1\notin A$. Therefore $z<z_1<y$ but $z\notin E$. This completes the proof.
  \end{enumerate}
\end{proof}
