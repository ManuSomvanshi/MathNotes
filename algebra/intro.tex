\section{Introduction}
\begin{definition}
  A group is a pair $(G, \cdot)$ where $G$ is a set and $\cdot: G\times G\to G$ is a binary operation such that:
  \begin{enumerate}
    \item $G$ is closed under the operation $\cdot$.
    \item $\cdot$ is associative.
    \item There exists $e\in G$ such that $a\cdot e = e\cdot a = a\ \forall\ a\in G$. This element is called the identity. 
    \item $\forall\ a\in G\ \exists\ b\in G$ such that $a\cdot b = b\cdot a =e$. $b$ is called the inverse of $a$ and is represented as $a^{-1}$. 
  \end{enumerate}
\end{definition}
\begin{proposition}
  The identity of a group $(G,\cdot)$ is unique.
\end{proposition}
\begin{proof}
  Let $e_1, e_2\in G$ be two identities. Since $e_1$ is an identity:
  \begin{align*}
    e_1\cdot e_2 = e_2 
  \end{align*}
  and since $e_2$ is an identity:
  \begin{align*}
    e_1\cdot e_2 = e_1
  \end{align*}
  Thus $e_1 = e_2$.
\end{proof}
\begin{proposition}
  The inverse of every element of the group $(G,\cdot)$ is unique.
\end{proposition}
\begin{proof}
  Let $a_1, a_2\in G$ both be inverse of $a$. Thus
  \begin{align*}
    a\cdot a_1 &= e\\
    \implies a_2\cdot a\cdot a_1 &= a_2\\
    \implies a_1 &= a_2.
  \end{align*}
  Hence the inverse is also unique. 
\end{proof}
\begin{proposition}
  Let $(G,\cdot)$ be a group and $x,y\in G$, then there exists $w,z\in G$ such that $x = w\cdot y$ and $x = y\cdot z$.
\end{proposition}
\begin{proof}
  Just choose $w = x\cdot y^{-1}$ and $z = y^{-1} x$. Then $w\cdot y = x \cdot y^{-1} \cdot y = x$, and $y\cdot z = y \cdot y^{-1} \cdot x = x$.
\end{proof}
\begin{notation}
  From now on the product between elements of any group will be written as $xy$ instead of $x\cdot y$. 
\end{notation}
\begin{proposition}
  The inverse of $(xy)^{-1} = y^{-1} x^{-1}$ where $x,y\in G$.
\end{proposition}
\begin{proof}
  Let $z\in G$ be the inverse of $xy$. Then:
  \begin{align*}
    xyz &= e\\
    \implies yz &= x^{-1}\\
    \implies z &= y^{-1} x^{-1}
  \end{align*}
  Also $zxy = y^{-1} x^{-1} xy = e$.
\end{proof}
\begin{definition}
  $I_n = \{1,...,n\}$ where $n\in \N$ and  $S_n = \{f:I_n \to I_n\ |\ \text{where $f$ is a bijection.}\ \}$.
\end{definition}
\begin{notation}
  Since the bijections on $I_n$ can be viewed as permutations we use the following notation: if $1 \to k_1, 2\to k_2,..., n \to k_n$ then,
  \begin{align*}
    f \doteq \begin{pmatrix}
      1 & 2 & ... & n\\
      k_1 & k_2 & ... & k_n
    \end{pmatrix}
  \end{align*}
  Also another notation commonly used is as follows. Let $f\in S_3$ be a bijection given by:
  \begin{align*}
    f \doteq \begin{pmatrix}
      1 & 2 & 3\\
      2 & 1 & 3
    \end{pmatrix}
  \end{align*}
  will be represented by $(12)(3)$ or just $(12)$.
\end{notation}
\begin{proposition}
  $(S_n, \circ)$ is a group (it's called the \textit{Permutation group}).
\end{proposition}
\begin{proof}
  Since the composition of two bijections is also a bijection $S_n$ is closed under the composition, and since the composition is associative property 2 is also satisfied. Since the inverse function of a bijection always exists. and is itself a bijection property 4 is satisfied. The identity map is obviously a bijection, thus it is in $S_n$.
\end{proof}
\begin{definition}
  The cardinality of a group is called the order. 
\end{definition}
\begin{proposition}\label{prop:fingrp}
  Let $(G,\cdot)$ be a finite group then $\forall\ a\in G\ \exists\ 0 \leq n\leq |G|$ such that $a^n = e$.
\end{proposition}
\begin{proof}
  Let's assume that such an $n$ does not exist. This means that each element $a, a^2, a^3,...$ is distinct, because if $a^n = a^m \implies a^{n-m} = e$. This contradicts the fact that $G$ is finite. If $n>|G|$ then it would contradict the fact that $G$ has $|G|$ number of elements. Thus $\exists\ n \leq |G|$ such that $a^n = e$.
\end{proof}
\begin{example}
  Let $n=3$, then $I_3 = \{1,2,3\}$. Let the points represent the nodes of an equilateral triangle, as in \cref{fig:triangle}. Now consider the bijections in $S_3$ such that the triangle remains unchanged. These bijections are rotations about the center of the circle i.e. $(123),\ (132)$, reflections about the medians i.e. $(12),\ (23),\ (13)$, and the identity map, $\text{id}_3$. All the symmetries of the triangle can be generated by composing $(123)$ and $(23)$ in different ways.
\end{example}
\begin{figure}[ht]
    \centering
    \incfig[0.7]{triangle}
    \caption{Geometric representation of $I_3$.}
    \label{fig:triangle}
\end{figure}
\begin{example}
  Similar to the previous example consider the $I_4$ to represent a square. Then all the possible bijections in $S_4$ which take the square to itself are the rotation: $(1234),\ (13)(24),\ (1432)$, the reflections about diagonals: $(13),\ (24)$, reflection along horizontal and vertical: $(14)(23),\ (12)(34)$, and the identity map: $\text{id}_4$. 
\end{example}
\begin{example}
  Let $\sim$ be an equivalence relation on $\Z$ given by $a \sim b \iff a\bmod n = b\bmod n$. The group formed by the quotient set, $\Z\bign/\sim$, under the operation $\oplus_n$ defined as $a\oplus_n b = (a+b)\bmod n$ is called the \textit{modulo-$n$ group} and often represented as $(\Z\bign/n\Z, +)$. For example the set $\Z\bign/3\Z = \{0,1,2\}$, where technically each element represents an equivalence class of integers with remainder $0,1,2$. 
\end{example}
\begin{example}
  Let $\otimes_n$ be an operation on $\Z\bign/n\Z$ such that $a\otimes_n b = (ab)\bmod n$. We will abuse notation and just write $ab$ instead of $a\otimes_n b$. Note that $\Z\bign/n\Z$ is not a group under $\otimes_n$. First reason is that $0$ does not have an inverse. Infact any number $a\in \Z\bign/n\Z$ such that $\gcd(a,n)\neq 1$ will not have an inverse. This can be shown by contradiction. If $\exists\ b\in \Z\bign/n\Z $ such that $ab = 1$ then $ab = kn + 1$. Now since $\gcd(a,n)$ divides $ab$ and $n$, but does not divide $1$, it divides LHS but not RHS leading to a contradiction. If we remove every element whose gcd with $n$ is not $1$ from $ \Z\bign/n\Z$ then we would get a group under $\otimes_n$. This group is represented by $((\Z\bign/n\Z)^*, \times)$. The cardinality of this group is given by the Euler totient function $\phi(n)$.
\end{example}
\begin{definition}
  A group is said to be \textit{abelian} if the product commutes.
\end{definition}
\begin{definition}
  A non-empty subset $H$ of a group $(G,\cdot)$ is said to be a subgroup if $(H,\cdot\big|_{H\times H})$ is a group. 
\end{definition}
\begin{proposition}\label{prop:subgrp}
  A non-empty subset $H$ of a group $(G,\cdot)$ is a subgroup iff it is closed under $\cdot\big|_{H\times H}$ and if $a\in H$ then $a^{-1}\in H$. 
\end{proposition}
\begin{proof}
  $(\implies)$ If $H$ is assumed to be a subgroup then by definition it is a group and thus is closed, and an inverse exists for each element.\\

  $(\impliedby)$ If $H$ is closed and for each $a\in H$ $a^{-1}\in H$ then definitely $e\in H$ since $aa^{-1} = e$. From the fact that $G$ is a group it can be deduced that $\cdot$ is associative and that $ae = ea = a$ and $aa^{-1} = a^{-1}a$.
\end{proof}
\begin{definition}
  A group $(G,\cdot)$ is said to be cyclic if $G=\{a^n\ |\ \forall\ n\in \Z\}$.
\end{definition}
\begin{example}
  The group $(\Z, +)$ is a cyclic group since $\Z = \{1^n\ |\ \forall\ n\in \Z\}$. This is because any element $a\in \Z$ can be written either as the sum $1+...+1$ or $(-1)+...+(-1)$.
\end{example}
\begin{proposition}
  Every non-empty finite subset of a group $(G,\cdot)$ that is closed under $\cdot$ is a subgroup of $G$. 
\end{proposition}
\begin{proof}
  Let $H\subset G$ be non-empty and closed under $\cdot$. By non-emptyness there is some element $a\in H$. Since $H$ is closed, all the powers of $a$ must be in $H$ as well. Since $H$ is finite, using a similar argument as in \cref{prop:fingrp}, there exists an $n\leq |H|$ such that $a^n =e$. This means that $a^{-1} = a^{n-1}$. Thus by $\cref{prop:subgrp}$ $H$ is a subgroup of $G$.
\end{proof}
\begin{remark}
  As a direct result of the above proposition one can show that every closed subset of a finite group is a subgroup.
\end{remark}
\begin{proposition}\label{prop:Zcycle}
  Every subgroup of $(\Z,+)$ is cyclic.
\end{proposition}
\begin{proof}
  Let $H\subset \Z$ be a subgroup. In the case $H=\{0\}$, $H$ is cyclic. Now consider $H$ to be any non-trivial subgroup. Due to closure if $x\in H$ then $-x\in H$, thus there exists positive integers in $H$. Let $d$ be the smallest positive integer in $H$ and let $n\in H$. Using the division algorithm one can write $n = qd + m$ where $0\leq m\leq d$. Again using closure since $d^q = qd \in H \implies d^{-q} = -qd \in H$. Thus $m = n - qd$. Since by definition $d$ is the smallest positive number the only way to avoid a contradiction is $m=0$. Thus $n=qd = d^q$ and $H$ is cyclic.
\end{proof}
\begin{proposition}\label{prop:Gcycle}
  Every subgroup of a cyclic group is cyclic.
\end{proposition}
\begin{proof}
  Let $(G,\cdot)$ be a cyclic group and $H\subset G$ be a non-trivial subgroup (claim is obviously true for trivial subgroup). Let $x$ be the generator for $G$. Since $H$ will only contain powers of $x$ define a set $K = \{n\ |\ x^n \in H\}$. If $n,m\in K$ then $x^n,x^m \in H\implies x^{n+m}\in H \implies n+m \in K$. Also If $n\in K$ then $x^n\in H\implies x^{-n} \in H\implies -n\in K$. Thus $(K,+)$ is a subgroup of $(\Z,+)$. By \cref{prop:Zcycle} $K$ is cyclic. If $d$ generates $K$ then $x^d$ generates $H$ since $x^n \in H\implies n\in K \implies n=qd \implies x^n = (x^d)^q$. 
\end{proof}
\begin{definition}
  Let $X\subset G$ where $(G,\cdot)$ is a group. Then $X$ is said to generate $G$ if $G = \{x_1^{n_1}...x_k^{n_k}\ |\ \forall\ x_i\in X,\ n_i\in\Z\}$. This is denoted by $G =\langle X \rangle$.
\end{definition}
\begin{definition}
  The order of an element $a$ of the group $(G,\cdot)$ is defined to be the cardinality of the subgroup generated by $a$. The order is denoted by $ \mathcal{O}(a)$.
\end{definition}
\begin{proposition}
  Let $a\in G$ where $(G,\cdot)$ is a group. Then the order of $a$ is $k$ iff $k$ is the smallest positive integer such that $a^k =e$.
\end{proposition}
\begin{proof}
  $(\implies)$ Assuming that the subgroup generated by $a$ has $k$ elements, i.e. $\{e,a,a^2,...,a^{k-1}\}$. By closure $a^k$ must be identified with one of the elements in $ \langle a \rangle$. If $a^k = a^n$ where $1\leq n\leq k$ then by cancelation $a^{k-1} = a^{n-1}$, implying that $ \langle a \rangle$ has cardinality less than $k$, which contradicts our assumption. Thus the only remaining possibility is that $a^k = e$. Moreover since all $a^i$ $0 \leq i\leq k-1$ are distinct it follows that $k$ is the smallest positive integer such that $a^k =1$.\\

  $(\impliedby)$ Assuming that $k$ is the smallest positive integer such that $a^k =1$, the subgroup $ \langle a \rangle = \{e, a, a^2, ..., a^{k-1}\}$. This follows from the argument that incase $a^n = a^m$ where $0<m<n<k$ then $a^{n-m} = e$ contradicting the fact that $k$ is the smallest such number. Thus all $a^m$ $0\leq m<k$ are distinct forming a subgroup of $k$.
\end{proof}
\begin{proposition}
  If $a,b\in G$ such that $ab= ba$ then $ \mathcal{O}(ab) = \mathcal{O}(a) \mathcal{O}(b)$.
\end{proposition}
\begin{proof}
  Let $ \mathcal{O}(a) = n$ and $ \mathcal{O}(b) = m$ and without loss of generality assume that $n\leq m$. Consider the subgroup $ \langle ab \rangle$. Since $ab\in \langle ab \rangle$ the power $(ab)^n = b^n a^n = b^n \in \langle ab \rangle$. Further it follows $b^nb^{m-n+1} = b^{m+1} = b\in \langle ab \rangle$. Similarly it can be shown that $a\in \langle a \rangle$. Thus $ \langle ab \rangle = \{a^i b^j\ |\ 0 \leq i < n,\ 0\leq j<m\}$ where I have used the fact that $ab=ba$ (otherwise we would have additional terms like $aba$). Thus the number of elements in $ \langle ab \rangle$ is $nm$.
\end{proof}
