\section{Introduction}
\begin{definition}
  A group is a pair $(G, \cdot)$ where $G$ is a set and $\cdot: G\times G\to G$ is a binary operation such that:
  \begin{enumerate}
    \item $G$ is closed under the operation $\cdot$.
    \item $\cdot$ is associative.
    \item There exists $e\in G$ such that $a\cdot e = e\cdot a = a\ \forall\ a\in G$. This element is called the identity. 
    \item $\forall\ a\in G\ \exists\ b\in G$ such that $a\cdot b = b\cdot a =e$. $b$ is called the inverse of $a$ and is represented as $a^{-1}$. 
  \end{enumerate}
\end{definition}
\begin{proposition}
  The identity of a group $(G,\cdot)$ is unique.
\end{proposition}
\begin{proof}
  Let $e_1, e_2\in G$ be two identities. Since $e_1$ is an identity:
  \begin{align*}
    e_1\cdot e_2 = e_2 
  \end{align*}
  and since $e_2$ is an identity:
  \begin{align*}
    e_1\cdot e_2 = e_1
  \end{align*}
  Thus $e_1 = e_2$.
\end{proof}
\begin{proposition}
  The inverse of every element of the group $(G,\cdot)$ is unique.
\end{proposition}
\begin{proof}
  Let $a_1, a_2\in G$ both be inverse of $a$. Thus
  \begin{align*}
    a\cdot a_1 &= e\\
    \implies a_2\cdot a\cdot a_1 &= a_2\\
    \implies a_1 &= a_2.
  \end{align*}
  Hence the inverse is also unique. 
\end{proof}
\begin{proposition}
  Let $(G,\cdot)$ be a group and $x,y\in G$, then there exists $w,z\in G$ such that $x = w\cdot y$ and $x = y\cdot z$.
\end{proposition}
\begin{proof}
  Just choose $w = x\cdot y^{-1}$ and $z = y^{-1} x$. Then $w\cdot y = x \cdot y^{-1} \cdot y = x$, and $y\cdot z = y \cdot y^{-1} \cdot x = x$.
\end{proof}
\begin{notation}
  From now on the product between elements of any group will be written as $xy$ instead of $x\cdot y$. 
\end{notation}
\begin{proposition}
  The inverse of $(xy)^{-1} = y^{-1} x^{-1}$ where $x,y\in G$.
\end{proposition}
\begin{proof}
  Let $z\in G$ be the inverse of $xy$. Then:
  \begin{align*}
    xyz &= e\\
    \implies yz &= x^{-1}\\
    \implies z &= y^{-1} x^{-1}
  \end{align*}
  Also $zxy = y^{-1} x^{-1} xy = e$.
\end{proof}
\begin{definition}
  $I_n = \{1,...,n\}$ where $n\in \N$ and  $S_n = \{f:I_n \to I_n\ |\ \text{where $f$ is a bijection.}\ \}$.
\end{definition}
\begin{notation}
  Since the bijections on $I_n$ can be viewed as permutations we use the following notation: if $1 \to k_1, 2\to k_2,..., n \to k_n$ then,
  \begin{align*}
    f \doteq \begin{pmatrix}
      1 & 2 & ... & n\\
      k_1 & k_2 & ... & k_n
    \end{pmatrix}
  \end{align*}
  Also another notation commonly used is as follows. Let $f\in S_3$ be a bijection given by:
  \begin{align*}
    f \doteq \begin{pmatrix}
      1 & 2 & 3\\
      2 & 1 & 3
    \end{pmatrix}
  \end{align*}
  will be represented by $(12)(3)$ or just $(12)$.
\end{notation}
\begin{proposition}
  $(S_n, \circ)$ is a group (it's called the \textit{Symmetric group}).
\end{proposition}
\begin{proof}
  Since the composition of two bijections is also a bijection $S_n$ is closed under the composition, and since the composition is associative property 2 is also satisfied. Since the inverse function of a bijection always exists. and is itself a bijection property 4 is satisfied. The identity map is obviously a bijection, thus it is in $S_n$.
\end{proof}
\begin{definition}
  The cardinality of a group is called the order. 
\end{definition}
\begin{proposition}
  Let $(G,\cdot)$ be a finite group then $\forall\ a\in G\ \exists\ 0 \leq n\leq |G|$ such that $a^n = e$.
\end{proposition}
\begin{proof}
  Let's assume that such an $n$ does not exist. This means that each element $a, a^2, a^3,...$ is distinct, because if $a^n = a^m \implies a^{n-m} = e$. This contradicts the fact that $G$ is finite. If $n>|G|$ then it would contradict the fact that $G$ has $|G|$ number of elements. Thus $\exists\ n \geq |G|$ such that $a^n = e$.
\end{proof}
\begin{example}
  Let $n=3$, then $I_3 = \{1,2,3\}$. Let the points represent the nodes of an equilateral triangle, as in \cref{fig:triangle}. Now consider the bijections in $S_3$ such that the triangle remains unchanged. These bijections are rotations about the center of the circle i.e. $(123),\ (132)$, reflections about the medians i.e. $(12),\ (23),\ (13)$, and the identity map, $\text{id}_3$. All the symmetries of the triangle can be generated by composing $(123)$ and $(23)$ in different ways.
\end{example}
\begin{figure}[ht]
    \centering
    \incfig[0.7]{triangle}
    \caption{Geometric representation of $I_3$.}
    \label{fig:triangle}
\end{figure}
\begin{example}
  Similar to the previous example consider the $I_4$ to represent a square. Then all the possible bijections in $S_4$ which take the square to itself are the rotation: $(1234),\ (13)(24),\ (1432)$, the reflections about diagonals: $(13),\ (24)$, reflection along horizontal and vertical: $(14)(23),\ (12)(34)$, and the identity map: $\text{id}_4$. 
\end{example}
\begin{example}
  Let $\sim$ be an equivalence relation on $\Z$ given by $a \sim b \iff a\bmod n = b\bmod n$. The group formed by the quotient set, $\Z\bign/\sim$, under the operation $\oplus_n$ defined as $a\oplus_n b = (a+b)\bmod n$ is called the \textit{modulo-$n$ group} and often represented as $(\Z\bign/n\Z, +)$. For example the set $\Z\bign/3\Z = \{0,1,2\}$, where technically each element represents an equivalence class of integers with remainder $0,1,2$. 
\end{example}
\begin{example}
  Let $\otimes_n$ be an operation on $\Z\bign/n\Z$ such that $a\otimes_n b = (ab)\bmod n$. We will abuse notation and just write $ab$ instead of $a\otimes_n b$. Note that $\Z\bign/n\Z$ is not a group under $\otimes_n$. First reason is that $0$ does not have an inverse. Infact any number $a\in \Z\bign/n\Z$ such that $\gcd(a,n)\neq 1$ will not have an inverse. This can be shown by contradiction. If $\exists\ b\in \Z\bign/n\Z $ such that $ab = 1$ then $ab = kn + 1$. Now since $\gcd(a,n)$ divides $ab$ and $n$, but does not divide $1$, it divides LHS but not RHS leading to a contradiction. If we remove every element whose gcd with $n$ is not $1$ from $ \Z\bign/n\Z$ then we would get a group under $\otimes_n$. This group is represented by $((\Z\bign/n\Z)^*, \times)$.
\end{example}
\begin{definition}
  A group is said to be \textit{abelian} if the product commutes.
\end{definition}
