\section{Free Groups}
Let $X$ be any set. We wish to construct a group using $X$. We can do this the following way:
\begin{enumerate}
  \item If $X = \emptyset$ then the group $F = \{e\}$.
  \item For non-empty sets, first choose a set, denoted $X^{-1}$, which is disjoint from $X$ and $|X| = |X^{-1}|$ (for infinite sets the cardinality should be same).
  \item Choose a bijection $f:X\to X^{-1}$. Denote the $f(x)$ by $x^{-1}$.
  \item Find a set disjoint from $X\cup X^{-1}$ which has cardinality $1$. Call the element of this set $1$.
\end{enumerate}
\begin{definition}
  A word on $X$ is a sequence in $X\cup X^{-1}\cup {1}$, $(a_1,a_2,...)$, such that $\exists\ N$ such that:
  \begin{align*}
    n>N \implies a_n = 1.
  \end{align*}
\end{definition}
\begin{definition}
  The constant sequence $(1,1,1,...)$ is called the empty word. We denote it by $1$ itself.
\end{definition}
\begin{definition}
  A reduced word is a word such that:
  \begin{enumerate}
    \item If $a_n =x$ then $a_{n+1} \neq x^{-1}$ and vice versa.
    \item If $a_k =1$ then $a_i = 1,\ \forall\ i\geq k$.
  \end{enumerate}
\end{definition}
\begin{notation}
  Every reduced word is of the form $(x_1^{n_1},...,x_{k}^{n_k},1,1,1,...)$ where $x_i\in X$ and $k\in \N$ and $n_i = \pm 1$. Thus we will formally write reduced words as $x_1^{n_1}...x_k^{n_k}$. Let $F(X)$ denote the set of all reduced words of $X$.
\end{notation}
\begin{definition}
  Define the product of two reduced words as:
  \begin{align*}
    x_1^{n_1}...x_k^{n_k}y_1^{m_1}...y_j^{m_k}
  \end{align*}
  where we remove every occurence of terms of the form $xx^{-1}$ or $x^{-1}x$. If all occurences are such, then the product is the empty word $1$. If $x$ is a reduced word, we define $x1 = 1x = x$.
\end{definition}
\begin{proposition}
  The set $F(X)$ is a group under the above product.
\end{proposition}
\begin{proof}
  The identity of the group is clearly the empty word. The inverse of a reduced word $x_1^{n_1}...x_k^{n_k}$ is $x_k^{-n_k}...x_1^{-n_1}$. By definition closure is ensured. The only hard part is checking associativity.\\

  To prove associativity, for each $x\in X$ and $n = \pm 1$ define $|x^n|: F\to F$ as:
  \begin{align*}
    1 &\mapsto x^n\\
    x_1^{n_1}...x_k^{n_k} &\mapsto \begin{cases}
      x^nx_1^{n_1}...x_k^{n_k},\ \text{if}\ x^n \neq x_1^{-n_1}\\
      x_2^{n_2}...x_k^{n_k}, \text{otherwise}. 
    \end{cases}
  \end{align*}
  Clearly $|x^n|$ is a bijection with inverse $|x^{-n}|$. Let $F_0$ be the group generated by the set $\{|x|\ |\ x\in X\}$ under the composition of bijection. Consider the map $\p: F(X)\to F_0$ given by $1\to \text{id}_F$ and $x_1^{n_1}...x_k^{n_k}\mapsto |x_1^{n_1}|...|x_k^{n_k}|$. Clearly $\p$ is surjective with the additional property that $\p(xy) = \p(x)\p(y)$. Since the composition of bijections is associative the preimage of the products in $F[X]$ will also be associative. Moreover $\p$ is a group isomorphism between $F_0$ and $F$. Also since the preimage of $\{|x|\ |\ x\in X\}$ is simply $X$, so we get that $F(x) = \langle X \rangle$.
\end{proof}
\begin{theorem}
  Let $F(X)$ be the free group of $X$ and let $i:X\to F(X)$ be the inclusion map. Let $G$ be a group and $f:X\to G$ be some function. Then there exists a unique group homomorphism $\tilde{f}$ such that the following diagram commutes:
  \[
    \begin{tikzcd}[sep=huge]
      X \arrow{r}{i} \arrow[swap]{dr}{f} & F(X) \arrow{d}{\tilde{f}} \\ & G 
    \end{tikzcd}
  \]
\end{theorem}
\begin{proof}
  Let $\tilde{f}(x_1^{n_1}...x_k^{n_k}) = f(x_1)^{n_1}...f(x_k)^{n_k}$ and $\tilde{f}(1) = e$, where $n_i = \pm 1$. Since $f(x_i)$ are elements of $G$, $f(x_i)^{n_i}$ are well defined.
  \begin{enumerate}
    \item Since
      \begin{align*}
        f(w_1w_2) = f(w_1)f(w_2)
      \end{align*}
      it follows that $\tilde{f}$ is a homomorphism.
    \item Let $x\in X$. Then $i(x)$ is the reduced word $x$ in $F(X)$. Then $\tilde{f}(x) = f(x)$. Thus the diagram commutes.
    \item Let $g:F(X)\to G$ be another homomorphism such that $g\circ i = f$. Since $g$ is a homomorphism $g(1) = e$ and $g(x^{-1}) = g(x)^{-1}$ for $x\in X$. Thus
      \begin{align*}
        g(x_1^{n_1}...x_{k}^{n_k}) &= g(x_1^{n_1})...g(x_k^{n_k}) = g(x_1)^{n_1}...g(x_k)^{n_k} = g\circ i(x_1)^{n_1}...g\circ i(x_k)^{n_k} = f(x_1)^{n_1}...f(x_k)^{n_k}\\ &= \tilde{f}(x_1^{n_1}...x_k^{n_k}).
      \end{align*}
      This means that $\tilde{f}$ is unique.
  \end{enumerate}
\end{proof}
\begin{corollary}
  Every group $G$ is a quotient of the free group.
\end{corollary}
\begin{proof}
  Let $G$ be a group and $X$ be the set of generators of $G$. Let $j:X\to G$ be the restriction of identity automorphism of $G$. Then the following diagram commutes due to the above theorem:
  \[
    \begin{tikzcd}[sep=huge]
      X \arrow{r}{i} \arrow[swap]{dr}{j} & F(X) \arrow{d}{\tilde{f}} \\ & G 
    \end{tikzcd}
  \]
  where $\tilde{f}$ is a unique homomorphism that takes $x\mapsto x\in G$. Since $G=\langle X \rangle$ we get that $\tilde{f}$ must be a sujection. Then by first isomorphism theorem,
  \begin{align*}
    G \simeq F(X)\bign/\Ker_{\tilde{f}}
  \end{align*}
\end{proof}
\begin{definition}
  Let $X$ be a set and $F(X)$ be the free group of $X$. A group $G$ is said to be defined by the generators $x\in X$ and relations $y\in Y$ if $G\simeq F(X)\bign/N$ where $N$ is a normal subgroup of $F(X)$, and $Y$ generates $N$. One says that $(X|Y)$ is the presentation of $G$.
\end{definition}
\begin{proposition}
  Let $G=(X|Y)$ and $H=(X|Y')$ where $Y\subset Y'$. Then $H$ is isomorphic to a quotient of $G$.  
\end{proposition}
\begin{proof}
  Since $Y\subset Y'$, it follows that $\langle Y \rangle \subset \langle Y' \rangle$. By definition the groups generated by $Y,Y'$ are normal. We know that $G \simeq F(X)\bign/\langle Y \rangle$ and $H\simeq F(X)\bign/ \langle Y' \rangle$Thus by third isomorphism thoerem:
  \begin{align*}
    G\bign/\langle Y' \rangle\bign/\langle Y \rangle \simeq F(X)\bign/\langle Y \rangle\bign/\langle Y' \rangle\bign/\langle Y \rangle \simeq F(X)\bign/\langle Y' \rangle \simeq H.
  \end{align*}
\end{proof}
