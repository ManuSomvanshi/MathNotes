\section{Caratheodory Theorem}
 Until now we have shown that extension $\nu$ of $\sigma-$additive measure $\mu$ on semi-algebra $ \mathscr{S}$ is also $\sigma-$additive on algebra $ \mathscr{A}$ generated by $ \mathscr{S}$. The goal of this section is to show that the extension $\pi: \mathscr{F}\to \R^+$, where $ \mathscr{F}$ is the $\sigma-$algebra generated by $ \mathscr{S}$ is $\sigma-$additive and unique. In order to do this we follow the following steps:
 \begin{enumerate}
   \item Define a $\pi^*: \mathfrak{P} (\Omega) \to \R^+$ and show that it is something called an \textit{outer measure}.
   \item Define a class $ \mathscr{M} \subset \mathfrak{P} (\Omega)$, and show that it is a $\sigma-$algebra.
   \item Show that $ \mathscr{A} \subset \mathscr{M}$. This has the implication that $ \mathscr{F} \subset \mathscr{M}$.
   \item Show that $\pi^*|_{ \mathscr{M} }$ is $\sigma-$additive and $\pi^*|_{ \mathscr{A} } = \nu$. Hence $\pi^*|_{ \mathscr{M} }$ is an extension.
   \item Finally show that this extension is unique.
 \end{enumerate}
 \begin{definition}
   Let $A \subset \Omega$ for some set $\Omega$. Then the collection $\{E_i\subset \Omega\ |\ i\geq 1\}$ is said to be a covering of $A$ if $A\subset \bigcup_{j\geq 1} E_j$. Note that at least one covering exists for every subset and that is $\{\Omega\}$.
 \end{definition}
 \begin{definition}
   Let $\pi^*: \mathfrak{P} (\Omega) \to \R^+$ for some set $\Omega$ defined in the following way: let $A \subset \Omega$ and let $\{E_i\in \mathscr{A}\ |\ i\geq 1\}$ be a covering of $A$ then
   \[\pi^*(A) = \inf_{\{E_i\}} \sum_{i\geq 1} \nu(E_i)\]
   This is to be read as infimum of $\sum_{i\geq 1} \nu(E_1)$ over all coverings of $A$ which are in the algebra $ \mathscr{A}$. 
 \end{definition}
 \begin{definition}
   Let $ \mathscr{C}$ be a class of subsets of $\Omega$ such that $\emptyset \in \mathscr{C}$, and let $\mu: \mathscr{C} \to \R^+$ be a function such that:
   \begin{enumerate}
     \item $\mu(\emptyset) = 0$,
     \item $\mu$ is monotone, i.e. $E\subset F$ where $E,F\in \mathscr{C}$ $\implies \mu(E)\leq \mu(F)$,
     \item $\mu$ is sub-additive, i.e. $E \in \mathscr{C}$ and $\{E_i\in \mathscr{C}\ |\ i\geq 1\}$ is a covering of $E$ then $\mu(E) \leq \sum_{i\geq 1} \mu(E_i)$.
   \end{enumerate}
   Then $\mu$ is said to be an outer measure. 
 \end{definition}
 \begin{proposition}\label{pro:outer}
   The function $\pi^*$ as defined above is an outer measure.
 \end{proposition}
 \begin{proof}
   Since $\emptyset \subset \Omega$, and it is a subset of every possible covering, clearly for the covering $\{E_i = \emptyset\ |\ \forall\ i\geq 1\}$,
   \[\sum_{i\geq 1} \nu(E_i) = 0\]
   and hence $\pi^*(\emptyset) = 0$.\\

   Let $E\subset F$ where $E,F \in \mathscr{C}$. Let $\{F_j\ |\ j\geq 1\}$ be a covering of $F$. Since $E\subset F$ any covering of $F$ is also a covering of $E$. If $E_j = F\cap F_j$ then $E_j \subset F_j$ and $\bigcup_{j\geq 1}E_j = F \supset E$. Hence $\{E_j\}$ is a covering $E$. Since $\nu$ is a $\sigma-$additive measure,
   \[\nu(E_j) \leq \nu(F_j)\ \text{and hence,}\ \sum_{i\geq 1}\nu(E_i) \leq \sum_{i\geq 1}\nu(F_i).\]
   Since for every covering of $F$ a covering of $E$ can be constructed in the above manner such that the above inequality is true, hence $\pi^*$ is monotone. \\

   For sub-additivity let $E \subset \Omega$ and let $\{E_i \in \mathscr{A}\ |\ i\geq 1\}$ be a covering of $E$. In the case when $\pi^*(E_i) = \infty$, clearly $\pi^*(E) \leq \pi^*(E_i)$. In the case when $\pi^*(E_i) < \infty\ \forall\ i\geq 1$, for each $\epsilon>0$ we can find a covering of $E_i$, say $\{F_{ij}\in \mathscr{A}\ |\ j\geq 1\}$ such that,
   \[\pi^*(E_i) \leq \sum_{j\geq 1} \nu(F_{ij}) \leq \pi^*(E_i) + \f{\epsilon}{2^i}\]
   Hence,
   \[\sum_{i\geq 1}\pi^*(E_i) \leq \sum_{j\geq 1} \nu(\bigcup_{j\geq 1}F_{ij}) = \sum_{i\geq 1}\nu(E_i)\leq \sum_{i\geq 1}\pi^*(E_i) + \epsilon\]
   Using $\pi^*(E) \leq \sum_{i\geq 1}\nu(E_i)$, we get
   \[\pi^*(E) \leq \sum_{i\geq 1}\nu(E_i) \leq \sum_{i\geq 1}\pi^*(E_i) + \epsilon\]
   Since this is true for arbitrary $\epsilon$ we conclude that $\pi^*$ is sub-additive. 
 \end{proof}
 \begin{definition}
   A set $A\subset\Omega$ is said to be measurable if $\forall\ E\subset \Omega$,
   \[\pi^*(E) = \pi^*(E\cap A) + \pi^*(E\cap A^c)\]
   Define $ \mathscr{M} $ to be the set of all measurable subsets of $ \Omega$.
 \end{definition}
 \begin{remark}
   Using sub-additivity of $\pi^*$ it is possible to prove that
   \[\pi^*(E) \leq \pi^*(E\cap A) + \pi^*(E\cap A^c)\]
   since $E = (E\cap A)\cup (E\cap A^c)$. Hence showing that $A$ is measurable just boils down to showing
   \[\pi^*(E) \geq \pi^*(E\cap A) + \pi^*(E\cap A^c)\]
 \end{remark}
 \begin{proposition}
   The alegebra $ \mathscr{A}$ of subsets of $\Omega$ is a subset of $ \mathscr{M}$.
 \end{proposition}
 \begin{proof}
   Let $A\in \mathscr{A}$ and let $E\in\Omega$. If we can show that $A$ is measurable, we prove the proposition. Let $\{E_i \in \mathscr{A}\}$ be a covering of $E$, and let $\e>0$. In the case when $\pi^*(E_i) = \infty$ even for a single $i$, it is clear that the inequality
   \begin{align}
     \pi^*(E) \geq \pi^*(E\cap A) + \pi^*(E\cap A^c)
   \end{align}
   holds. In the case when $\pi^*(E_i) < \infty$ forall $i\geq 1$ let $\e>0$. Then
   \[\pi^*(E) \leq \sum_{i\geq 1}\nu(E_i) \leq \pi^*(E) + \e \]
   Since $E\cap A \subset \bigcup_{i\geq 1} E_i\cap A$,
   \[\pi^*(E\cap A) \leq \sum_{i \geq 1} \nu(E_i \cap A)\]
   Using similar arguments for $A^c$
   \[\pi^*(E\cap A^c) \leq \sum_{i \geq 1} \nu(E_i \cap A^c)\]
   Adding these two inequalities
   \[\pi^*(E\cap A^c) + \pi^*(E\cap A) \leq \sum_{i \geq 1} \nu(E_i)\]
   Here I have used the additivity of $\nu$ since $E_i\cap A$ and $E_i\cap A^c$ are in the algebra. Using inequality (1):
   \[\pi^*(E\cap A^c) + \pi^*(E\cap A) \leq \sum_{i \geq 1} \nu(E_i) \leq \pi^*(E) + \e\]
   Since this is true for arbitrary $\e$, we have
   \[\pi^*(E\cap A^c) + \pi^*(E\cap A)  \leq \pi^*(E) \]
   Hence we have shown that $A$ is measurable, completing the proof.
 \end{proof}
 \begin{proposition}
   $ \mathscr{M}$ is a $\sigma-$algebra.
 \end{proposition}
 \begin{proof}
   Since every algebra is a subset of $ \mathscr{M}$ clearly $\Omega \in \mathscr{M}$. Also it is easy to see that if $A\in \mathscr{M}$ then $A^c \in \mathscr{M}$ since replcaing $A$ by $A^c$ in the condition of measurable set does not change the inequality.\\
   The only condition that remains to be checked is closure under countable union. First consider the finite case. Let $A, B \in \mathscr{M}$. We are required to show that $\forall\ E\subset \Omega$,
   \[\pi^*(E) \geq \pi^*(E\cap (A\cup B)) + \pi^*(E\cap (A\cup B)^c).\]
   Since
   \[\pi^*(E) = \pi^*(E\cap A) + \pi^*(E-A),\]
   \[\pi^*(E) = \pi^*(E\cap B) + \pi^*(E-B)\]
   Thus
   \begin{align*}
     \pi^*(E-A) &= \pi^*((E-A)\cap B) + \pi^*((E-A)-B)\\
             &= \pi^*(E\cap A^c \cap B) + \pi^*(E-(A\cup B)^c),
   \end{align*}
   implying that
   \begin{align*}
     \pi^*(E) &= \pi^*(E\cap A) +\pi^*(E\cap A^c \cap B) + \pi^*(E-(A\cup B)^c)\\
           &\geq \pi^*(E\cap A \cup B)+ \pi^*(E-(A\cup B)^c).
   \end{align*}
   The final inequality comes from the sub-additivity of $\pi^*$ and the fact that $(E\cap A) \cup (E \cup A^c \cup B) = E \cap A \cup B$. Hence $A\cup B \in \mathscr{M}$. Now extending this to the countable case, let $A_j \in \mathscr{M}$, $A = \bigcup_{j\geq 1}A_j$, and $B_n = \bigcup_{j= 1}^{n} A_{j}$. Using closure under finite unions we can say that
   \[\pi^*(E) = \pi^*(E\cap B_n) + \pi^*(E-B_n)\]
   Since $B_n \subset A\implies E-B_n \supset E-A$. Hence,
   \[\pi^*(E) \geq \pi^*(E\cap B_n) + \pi^*(E-A)\]
   Define the sets $F_1 = A_1$, ..., $F_j = A_j - B_{j-1}$, ...; and observe that $F_j \in \mathscr{M}$, $A = \bigcup_{j\geq1} F_j$, and that these sets are pairwise disjoint. If we define $G_n = \bigcup_{j=1}^n F_j$, then using a similar logic as $B_n$
   \[\pi^*(E) \geq \pi^*(E\cap G_n) + \pi^*(E-A).\]
   Using induction one can show that
   \[\pi^*(E \cap \bigcup_{j=1}^n F_j) = \sum_{j=1}^n \pi^*(E\cap F_j).\]
   For $n=1$ it is obviously true. Assuming it to be true for some $n$,
   \begin{align*}
     \pi^*(E \cap \bigcup_{j=1}^{n+1} F_j) &= \pi^*(E \cap \bigcup_{j=1}^{n+1} F_j \cap F_{n+1}) + \pi^*(E \cap \bigcup_{j=1}^{n+1} F_j \cap F_{n+1}^c)\\
     &= \pi^*(E\cap F_{n+1}) + \pi^*(E\cap \bigcup_{j=1}^n F_j)\\
     &= \pi^*(E\cap F_{n+1}) + \sum_{j=1}^{n}\pi^*(E\cap F_j)\\
     &= \sum_{j=1}^{n+1}\pi^*(E\cap F_j).
   \end{align*}
   Hence using this property,
   \[\pi^*(E) \geq \pi^*(E\cap G_n) + \pi^*(E-A) = \sum_{j=1}^n\pi^*(E\cap F_j) + \pi^*(E-A).\]
   Taking the limit $n\to \infty$,
   \[\pi^*(E) \geq \sum_{j\geq1}\pi^*(E\cap F_j) + \pi^*(E-A) \geq \pi^*(E\cap A) + \pi^*(E-A).\]
   The final inequality comes from sub-additivity of $\pi^*$. This completes the proof that $ \mathscr{M}$ is a $\sigma-$algebra. 
 \end{proof}
 \begin{remark}
   Since the algebra $ \mathscr{A}$ is a subset of $ \mathscr{M}$ and $ \mathscr{M}$ is a $\sigma-$algebra it follows that $ \mathscr{M}$ contains the $\sigma-$algebra generated by $ \mathscr{A}$. 
 \end{remark}
 \begin{proposition} 
   $\pi^*(A) = \nu(A)\ \forall\ A\in \mathscr{A}$.
 \end{proposition}
 \begin{proof}
   Let $A \in \mathscr{A}$. Consider the covering $\{A_1 = A,\ A_j = \emptyset\ j\geq 2\}$. Then $\pi^*(A) \leq \nu(A)$ by definition. The opposite inequality can be proved by constructing sets $F_1 = E_1$, $F_n = E_n - \bigcup_{j=1}^{n-1}E_j$, where $\{E_n\in \mathscr{A}\}$ is some covering of $A$. As discussed in the previous proof $F_j$ are pairwise disjoint. Since,
   \begin{align*}
     &A \subset \bigcup_{j\geq1}F_j \\
     \implies &A = \bigcup_{j\geq1}F_j\cap A\\
     \implies &\nu(A) = \nu(\bigcup_{j\geq1}F_j\cap A)\\
     \implies &\nu(A) = \sum_{j\geq1} \nu(F_j \cap A) \leq \sum_{j\geq1} \nu(E_j)
   \end{align*}
   The last inequality comes from the fact that $F_j \cap A \subset E_j$. This inequality shows that $\nu(A)$ is infact the infimum of the sum over all coverings of $A$. Hence $\pi^*(A) = \nu(A)$.  
 \end{proof}
 \begin{proposition}
   $\pi^*|_{ \mathscr{M} }$ is $\sigma-$additive.
 \end{proposition}
 \begin{proof}
   It is clear that $\pi(\emptyset) = 0$. Let $A_1,A_2,...\in \mathscr{M}$ be pairwise disjoint sets and let their union be $A$. Since $ \mathscr{M}$ is a $\sigma-$algebra $A\in \mathscr{M}$. Since we have already shown that for any pairwise disjoint sets $F_1, F_2,... \in \mathscr{M}$ and any $E\subset \Omega$
   \[\pi^*(E \cap \bigcup_{j=1}^n F_j) = \sum_{j=1}^n \pi^*(E\cap F_j).\]
   Letting $E=A$ and $F_j = A_j$,
   \[\pi^*(\bigcup_{j= 1}^nA_j) = \pi^*(A \cap \bigcup_{j=1}^n A_j) = \sum_{j=1}^n \pi^*(A\cap A_j) = \sum_{j=1}^n \pi^*(A_j).\]
   Since $\bigcup_{j=1}^n A_j \subset \bigcup_{j\geq 1} A_j$, using monotonicity of $\pi^*$
   \[\pi^*(A) \geq \sum_{j=1}^n\pi^*(A_j).\]
   Taking the limit
   \[\pi^*(A) \geq \sum_{j\geq1}\pi^*(A_j).\]
   Since using sub-additivity we already know that
   \[\pi^*(A) \leq \sum_{j\geq1}\pi^*(A_j)\]
   it follows that $\pi^*$ acting on $ \mathscr{M}$ is $\sigma-$additive. 
 \end{proof}
 \begin{definition}
   A set $\Omega$ is said to be $\sigma-$finite with respect to a function $\mu$ if there exists a sequence $E_1,E_2,... \subset \Omega$,  such that $E_j \uparrow \Omega \implies \mu(E_j)<\infty$.
 \end{definition}
 \begin{definition}
   A class $ \mathscr{G} \subset \mathfrak{P} (\Omega)$ is said to be a monotone class if all monotonic sequences of sets converge in $ \mathscr{G}$.
 \end{definition}
 \begin{proposition}
   If $ \mathscr{G}_\alpha$ where $\alpha\in I\subset\R$ are monotone classes then the intersection $\bigcap_{\alpha\in I} \mathscr{G}_\alpha$ is also a monotone class. 
 \end{proposition}
 \begin{proof}
   If $A_1,A_2...$ is any monotone sequence in $\bigcap_{\alpha \in I} \mathscr{G}_\alpha$ then it is in all $ \mathscr{G}_\alpha$ and hence converge in all $ \mathscr{G}_\alpha$.
 \end{proof}
 \begin{remark}
   Using this proposition one can define the smallest monotone class generated by some class $ \mathscr{C}$ as the intersection of all the monotone classes containing $ \mathscr{C}$.
 \end{remark}
 \begin{lemma}\label{lem:damn}
   Let $ \mathscr{A}$ be any algebra of subsets of $\Omega$, $ \mathscr{G}$ be the monotone class generated by $ \mathscr{A}$, and let $ \mathscr{F}$ be the $\sigma-$algebra generated by $ \mathscr{A}$. Then $ \mathscr{G} = \mathscr{F}$.
 \end{lemma}
 \begin{proof}
   Let $A_j \in \mathscr{F}$ monotonically increase (decrease) to $A$; since the countable union (intersection) of $A_j$ is in $ \mathscr{A}$ so $A$ must also be in $ \mathscr{F}$. Since $ \mathscr{A}\subset \mathscr{F}$ it follows that $ \mathscr{G}\subset \mathscr{F}$.\\

   All that remains to be shown is $ \mathscr{F} \subset \mathscr{G}$. If we can show that $ \mathscr{G}$ is an algebra, then for any $A_1, A_2,...\in \mathscr{G}$ let $B_i = \bigcup_{j=1}^i A_i \in \mathscr{G}$ (since if $ \mathscr{G}$ is an algebra it will be closed under finite union), it follows that $\bigcup_{j\geq 1} B_j = \bigcup_{j\geq 1}A_j \in \mathscr{G}$ (using the fact that $\mathscr{G}$ is monotone class). To prove that $ \mathscr{G}$ is an algebra define $A \in \mathscr{G}$ the class:
   \[ \mathscr{M} (A) =\{ M\in \mathscr{G}\ |\ A-M,\ M-A,\ A\cap M \in \mathscr{G}\}.\]
   Clearly $ \mathscr{M} (A) \subset \mathscr{G}$. Let $E_1\subset E_2 \subset ... \in \mathscr{M} (A)$ converge to some $E$. Then $E-A = \bigcup_{i\geq 1} (E_i - A)$, but since by definition $E_i-A \in \mathscr{G}$ and $E_i-A \subset E_{i+1} -A$ it follows that $E-A\in \mathscr{G}$ (since $ \mathscr{G}$ is monotone class). Similarly it can be shown that $A-E$ and $A\cap E$ are in $ \mathscr{G}$, and thus $E \in \mathscr{M} (A)$. The same argument can be used to show that if $E_1 \supset E_2 \supset... \in \mathscr{M} (A)$ converges to $E$ then $E \in \mathscr{M} (A)$, hence concluding that $ \mathscr{M} (A)$ is a monotone class. $ \mathscr{M} (A)$ is also symmetric in the sense that if $A \in \mathscr{M} (B) \iff B \in \mathscr{M} (A)$, because if $A-B, B-A, A\cap B \in \mathscr{G}$ then both $A \in \mathscr{M} (B)$ and $B \in \mathscr{M} (B)$.\\

   Let $A,B \in \mathscr{A}$ then we know that $A-B, B-A, A\cap B \in \mathscr{A}$. Hence $B \in \mathscr{M} (A)$ forall $A,B \in \mathscr{A}$, implying that $ \mathscr{A} \subset \mathscr{M} (A)\ \forall\ A \in \mathscr{A}$. Since $ \mathscr{G}$ is the smallest monotone class containing $ \mathscr{A}$, $ \mathscr{G} \in \mathscr{M} (A)\ \forall\ A\in \mathscr{A}$. Since $M \in \mathscr{A}$ forall $M \in \mathscr{G}$ using the symmetry it implies that $ A \in \mathscr{M} (M)$. Hence $ \mathscr{A} \in \mathscr{M}(M)$. Therefore $ \mathscr{G} = \mathscr{M} (M)$ forall $M \in \mathscr{G}$. This means that $ \mathscr{G}$ is closed under finite difference and intersection, proving that it is an algebra. This completes the proof.     
 \end{proof}
 \begin{theorem}[Uniqueness of Extension]
   Let $\mu_1, \mu_2: \mathscr{F} \to \R^+$ be $\sigma-$additive functions, where $ \mathscr{F}$ is the $\sigma-$algebra generated by algebra $ \mathscr{A}$ of a set $\Omega$ which is $\sigma-$finite with respect to $\mu_1$ and $\mu_2$ (with the additional condition that the finite sequence exists in $ \mathscr{A}$), be such that $\mu_1|_{ \mathscr{A}} = \mu_2|_{ \mathscr{A}}$ then $\mu_1 = \mu_2$. 
 \end{theorem}
 \begin{proof}
   Let $E_1, E_2,... \in \mathscr{A}$ be the sequence such that $\mu_1(E_n)<\infty$ and $\mu_2(E_n)<\infty$ for all $n$ and $E_n \uparrow \Omega$. This sequence is gauranteed by the $\sigma-$finiteness of $\Omega$. Define $ \mathscr{B}_n = \{E \in \mathscr{F}\ |\ \mu_1(E\cap E_n) = \mu_2(E\cap E_n)\}$. Clearly $ \mathscr{B}_n \subset \mathscr{F}$. If $E \in \mathscr{A}$ then $E\cap E_n \in \mathscr{A}$ and since $\mu_1|_{ \mathscr{A}} = \mu_2|_{ \mathscr{A}}$ it follows that $ \mathscr{A} \subset \mathscr{B}_n$. Let $A_1,A_1,...\in \mathscr{B}$ be a sequence monotonically converging to some $A$. Since
   \begin{align*}
     \mu_1(A_j\cap E_n) &= \mu_2(A_j\cap E_n)\\
     \implies \mu_1(A\cap E_n) &= \mu_2(A\cap E_n)
   \end{align*}
   where we have used \cref{lem:continuity} and the finiteness of $E_n$ in case of continuity from above. It follows that $ \mathscr{B}_n$ is a monotone class. Since it contains the algebra $ \mathscr{A}$ as well it must contain the monotone class generated by $ \mathscr{A}$. Using \cref{lem:damn} we can conclude that $ \mathscr{F}\subset \mathscr{B}_n$ and hence $ \mathscr{B}_n = \mathscr{F}$. Let $A\in \mathscr{F}$ then
   \begin{align*}
     \lim_{n\to \infty}\mu_1(A\cap E_n) &= \lim_{n\to\infty}\mu_2(A\cap E_n)\\
     \implies \mu_1(A) &= \mu_2(A)
   \end{align*}
   Therefore the extension is unique.
 \end{proof}
 As a result of this theorem we can conclude that the function $\pi^*: \mathscr{F}\to\R^+$ is a $\sigma-$additive extension  of the $\sigma-$additive measure $\nu: \mathscr{A}\to \R^+$ on the $\sigma-$algebra $ \mathscr{F}$ generated by the algebra $ \mathscr{A}$ and is uniquely determined. This is known as Caratheodory theorem. The formal statement of this theorem is:
 \begin{theorem}[Caratheodory Theorem]\label{thm:caratheodory}
   Let $ \mathscr{A}$ be an algebra, $\nu: \mathscr{A}\to \R^+$ be a $\sigma-$additive measure, and $ \mathscr{F}$ be the $\sigma-$algebra generated by $ \mathscr{A}$. Then there exists a unique $\sigma-$additive measure $\pi: \mathscr{M} \to \R^*$ such that $\pi|_{ \mathscr{A}} = \nu$. Explicity this measure is given by restricting the outer measure $\pi^*: \mathfrak{P} (\Omega) \to \R^+$,
   \[\pi^*(A) = \inf_{\{E_j\in \mathscr{A}\}} \sum_{j\geq 1} \nu(E_j),\ \text{where}\ \{E_j\}\ \text{is a covering of $A$}\]
   on $ \mathscr{M}$; i.e. $\pi = \pi^*|_{ \mathscr{M}}$. 
 \end{theorem}
