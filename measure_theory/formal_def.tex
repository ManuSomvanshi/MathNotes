\section{Formal Notion of Measure}
\begin{definition}
  A class of subsets, $\mathscr{A}$, of a set $\Omega$ is said to be a semi-algebra if:
  \begin{enumerate}
    \item $\Omega\in \mathscr{A}$,
    \item closed under finite intersections,
    \item The compliment of any set in $\mathscr{A}$ can be expressed as unions of finite pairwise disjoint sets in $\mathscr{A}$.
  \end{enumerate}
\end{definition}
\begin{definition}
  A class of subsets, $\mathscr{A}$, of a set $\Omega$ is said to be an algebra if:
  \begin{enumerate}
    \item $\Omega\in \mathscr{A}$,
    \item closed under finite intersections,
    \item Closed under compliment.
  \end{enumerate}
\end{definition}
\begin{definition}
  A class of subsets, $\mathscr{A}$, of a set $\Omega$ is said to be a $\sigma$-algebra if:
  \begin{enumerate}
    \item $\Omega\in \mathscr{A}$,
    \item closed under countable intersections,
    \item Closed under compliment.
  \end{enumerate}
\end{definition}
\begin{proposition}
  Let $\Omega$ be a set and $\mathscr{A}_\alpha\subset \mathfrak{P} (\Omega)$ be algebras, where $\alpha \in I$ (no assumptions have been made on $I$). Then
  \[\mathscr{A}= \bigcap_{\alpha\in I} \mathscr{A}_\alpha\]
  is also an algebra.
\end{proposition}
\begin{proof}
  Since $\Omega \in \mathscr{A}_\alpha,\ \forall\ \alpha\in I$, implies that $\Omega\in\mathscr{A}$. If $A_1,...,A_n \in \mathscr{A}$ then $A_1,...,A_n \in \mathscr{A}_\alpha$ for any $\alpha\in I$. Since $\mathscr{A}_\alpha$ is an algebra, it follows that $\bigcap_{j=1}^{n}A_j$ is in $\mathscr{A}_\alpha$ for any $\alpha\in I$; hence it is also in $\mathscr{A}$. If $A\in\mathscr{A}$ then it is in every $\mathscr{A}_\alpha$ and hence its compliment is in every $\mathscr{A}_\alpha$.  
\end{proof}
\begin{remark}
The above proposition also applies to $\sigma$-algebras as well. Essentially the same argument applies, just that instead of finite sets we have countable intersection, i.e. $n\to \infty$. To denote that something applies to both algebras and $\sigma$-algebras we use the notation $(\sigma-)$algebra.
\end{remark}
\begin{definition}
  A class $\mathscr{C}$ of subsets of set $\Omega$ is said to \textit{generate} an $(\sigma-)$algebra $\mathscr{A}$ if $\mathscr{C}\subset \mathscr{A}$ and if for any $(\sigma-)$alegbra $\mathscr{A}'\supset \mathscr{C}$ implies that $\mathscr{A}\subset \mathscr{A}'$.
\end{definition}
\begin{proposition}
  Every class $\mathscr{C}\subset \mathfrak{P} (\Omega)$ generates an $(\sigma-)$algebra.
\end{proposition}
\begin{proof}
  Let $\mathscr{A}_\alpha,\ \alpha\in I$ be all the $(\sigma-)$algebras which contain the class $\mathscr{C}$. Then we know that,
  \[\mathscr{A} = \bigcap_{\alpha\in I} \mathscr{A}_\alpha\]
  is also an $(\sigma-)$algebra, and it will contain $\mathscr{C}$. From the definition of intersection it follows that $\mathscr{A}\subset \mathscr{A}_\alpha$. Hence $\mathscr{A}$ is the $(\sigma-)$algebra generated by $\mathscr{C}$.
\end{proof}
\begin{lemma}\label{lem:generated}
  If $\mathscr{S}$ is a semi-algebra and $\mathscr{A}$ is the algebra generated by $\mathscr{S}$ then
  \[A\in \mathscr{A} \iff \exists\ \text{pairwise disjoint}\ E_1,...,E_n\in\mathscr{S}\ \text{such that}\ A = \bigcup_{j=1}^{n} E_j \]
\end{lemma}
\begin{proof}
  ($\impliedby$) Assuming that $A$ is finite union of disjoint sets $E_1,...,E_n\in \mathscr{S}$ we need to show that $A\in\mathscr{A}$. Since $E_1,...,E_n$ are in $\mathscr{S}$ it follows that they are also in $\mathscr{A}$. It further follows that the compliment of each $E_j\in \mathscr{A}$. Since
  \[\l(\bigcap_{j=1}^n E_j^c \r)^c = \bigcup_{j=1}^n E_j,\]
  and algebras are closed under finite intersections, $A\in \mathscr{A}$.
  \paragraph{($\implies$)} Let $\mathscr{B}$ be the class defined as:
  \[ \mathscr{B} = \{B\ |\ \text{where}\ B=\bigcup_{j=1}^n F_j,\ F_j \in \mathscr{S}\ \text{are pairwise disjoint.} \}\]
  If we can show that $ \mathscr{B}$ is an algebra containing $ \mathscr{S}$ then by definition of generated algebras $ \mathscr{A}\subset \mathscr{B}$. This shows that any element of $ \mathscr{A}$ can be expressed as a finite union of disjoint sets. Hence all that remains is to show that $ \mathscr{B}$ is an algebra containing $ \mathscr{S}$.
  \begin{enumerate}
    \item Clearly by the definition, any element of $ \mathscr{S}$ is also in $ \mathscr{B}$. Hence $ \mathscr{S} \subset \mathscr{B}$ and hence $\Omega \in \mathscr{B}$.
    \item Let $B_1,...,B_n\in \mathscr{B}$ then
      \begin{align*}
        \bigcap_{j=1}^n B_j &= \bigcap_{j=1}^n \bigcup_{i=1}^m F_{ji}\\
                 &= \bigcup_{i=1}^m \bigcap_{j=1}^n F_{ji},\ \text{using definition of $ \mathscr{B}$}\\
                 &= \bigcup_{i=1}^m E_i,\ \text{where,}\ E_i = \bigcap_{j=1}^n F_{ji}
      \end{align*}
      Since $ \mathscr{S}$ is closed under finite intersections, this shows that $ \mathscr{B}$ is closed under finite intersections.
    \item Let $B\in \mathscr{B}$. Then,
      \begin{align*}
        B^c &= \l( \bigcup_{i=1}^m F_i \r)^c\\
           &= \bigcap_{i=1}^m F_i^c\\
           &= \bigcap_{i=1}^m \bigcup_{j=1}^n E_{ij}\ \text{(using property 3 of semi-algebras)}\\
           &= \bigcup_{j=1}^n E_j,\ \text{where}\ E_j= \bigcap_{i=1}^m E_{ij}
      \end{align*}
      Since $ \mathscr{S}$ is closed under finite intersections, this shows that $ \mathscr{B}$ is closed under compliment. 
  \end{enumerate}
  $ \mathscr{B}$ is indeed an algebra containing $ \mathscr{S}$, hence completing our proof.
\end{proof}
\begin{definition}
  Let $ \mathscr{C}$ be a class of subsets of $\Omega$ such that $\emptyset \in \mathscr{C}$, and let $\mu: \mathscr{C}\to\R^+$ be a function such that:
  \begin{enumerate}
    \item $\mu(\emptyset) = 0$,
    \item If $E_1,...,E_n \in \mathscr{C}$ are pairwise disjoint and if $\bigcup_{j=1}^nE_j \in \mathscr{C}$ then $\mu(\bigcup_{j=1}^nE_j)=\sum_{j=1}^n\mu(E_j)$.
  \end{enumerate}
  then $\mu$ is said to be an additive measure.
\end{definition}
\begin{remark}
Observe that if we have a $A\in \mathscr{C}$ such that $\mu(A)<\infty$ then:
\begin{align*}
  \mu(A\cup \emptyset) &= \mu(A) + \mu(\emptyset)\\
  \mu(A) &= \mu(A) + \mu(\emptyset)\\
  \implies \mu(\emptyset) &= 0
\end{align*}
Hence the first condition is just a consequence of the second if a subset with finite measure exists. Secondly observe that if $E\subset F \in \mathscr{C}$ and $F-E\in \mathscr{C}$ then:
\begin{align*}
  \mu(E\cup F-E) = \mu(F) = \mu(E) + \mu(F-E)
\end{align*}
this means that $\mu(E)\leq \mu(F)$, the equality being true when $\mu(E) = \infty$. In the case where $\mu(E)<\infty$ we have the identity $\mu(F-E) = \mu(F)-\mu(E)$.
\end{remark}
\begin{example}
  Let $\Omega$ be any non-empty set and let $X_1,X_2,...\in \Omega$. Also let $a_1, a_2,...\geq0$ be some constants. Then define a measure $\mu: \mathscr{C}\subset (\mathfrak{P}(\Omega))\to \R^+$ as:
  \begin{align*}
    \mu(A) = \sum_{j\geq 1} a_j 1\{X_j\in A\}
  \end{align*}
  where,
  \begin{align*}
    1\{X_j\in A\} = \begin{cases}
      1,\ \text{if}\ X_j\in A\\
      0,\ \text{if}\ X_j\notin A
    \end{cases}
  \end{align*}
  It is easy to see that this measure is indeed additive. 
\end{example}
\begin{definition}
  Let $ \mathscr{C}$ be a class of subsets of $\Omega$ such that $\emptyset \in \mathscr{C}$, and let $\mu: \mathscr{C}\to\R^+$ be a function such that:
  \begin{enumerate}
    \item $\mu(\emptyset) = 0$,
    \item If $E_1,E_2,... \in \mathscr{C}$ are pairwise disjoint and if $\bigcup_{j\geq1}E_j \in \mathscr{C}$ then $\mu(\bigcup_{j\geq1}E_j)=\sum_{j\geq1}\mu(E_j)$.
  \end{enumerate}
  then $\mu$ is said to be a $\sigma-$additive measure.
\end{definition}
\begin{example}
  Let $\Omega = (0,1)$ and $ \mathscr{C} = \{(a,b]\ |\ 0\leq a < b < 1\ \}\cup\{\emptyset\}$. Define a function $\mu: \mathscr{C}\to \R^+$ as:
  \begin{align*}
    \mu(a,b] = \begin{cases}
      \infty,\ \text{if}\ a=0\\
      b-a,\ \text{if}\ a\neq 0
    \end{cases}
  \end{align*}
  Clearly since a subset with finite measure exists $\mu(\emptyset) = 0$. Also since,
  \[(a,b] = \bigcup_{j=1}^n (a_j,a_{j+1}],\ \text{where}\ a_1 = a \And a_n=b\] 
  when $a=0$, $a_1 = 0$ and hence applying the measure on both sides we get $\infty$. When $a\neq 0$, so are none of the $a_j$ and hence:
  \[\mu(a,b] = b-a = (a_2 - a_1] + ... + (a_n - a_{n-1}] = \sum_{j=1}^n\mu(a_j,a_{j+1}]\]
  Hence $\mu$ is additive. But it is possible to show that $\mu$ is not $\sigma-$additive. Consider for example the interval $(0,1/2]$, and let $x_1=1/2,x_2,...$ be a monotonic decreasing sequence in $(0,1)$ which converges to $0$. Then
  \[(0,1/2] = \bigcup_{j\geq 1} (x_{j+1}, x_j]\]
  Clearly $\mu(0,1/2] = \infty$, but $\mu(x_{j+1},x_j] = x_{j+1} - x_j$ which is finite.
\end{example}
\begin{definition}
  Let $ \mathscr{C}$ be a class of subsets of $\Omega$ and $\mu: \mathscr{C}\to \R^+$ be any set function. Then,
  \begin{enumerate}
    \item $\mu$ is said to be \textit{continuous from below} at $E\in \mathscr{C}$ if $\forall\ (E_n)_{n\geq 1}\in \mathscr{C}$, $E_n \uparrow E \implies \lim\mu(E_n) = \mu(E)$.
    \item $\mu$ is said to be \textit{continuous from above} at $E\in \mathscr{C}$ if $\forall\ (E_n)_{n\geq 1}\in \mathscr{C}$, $E_n \downarrow E$ and $\exists\ n_0$ such that $\mu(E_{n_0}) <\infty$ implies that $\lim\mu(E_n)= \mu(E)$. 
  \end{enumerate}
\end{definition}
\begin{remark}
  If the condition of existence of $n_0$ such that $\mu(E_{n_0})<\infty$ is removed then some unwanted cases arise. For example consider a measure on some class of $\R$. Consider the sequence of intervals $I_n = [n,\infty)$. Clearly $\bigcup_{n\geq 1} [n,\infty) = \emptyset$, but $\mu(\emptyset) = 0$ while $\mu(I_n) = \infty,\ \forall\ n$. This shows that no measure can be continuous from above on $\R$. This leads us to add the condition of existence of some set in the sequence which has finite measure. 
\end{remark}
\begin{lemma}\label{lem:continuity}
  Let $ \mathscr{A}$ be an algebra and let $\mu: \mathscr{A}\to \R^+$ be an additive measure, then:
  \begin{enumerate}
    \item $\mu$ is $\sigma-$additive $\implies$ $\mu$ is continuous.
    \item $\mu$ is continuous from below $\implies$ $\mu$ is $\sigma-$additive.
    \item $\mu$ is continuous from above at $\emptyset$ and $\mu$ is a finite measure $\implies$ $\mu$ is $\sigma-$additive.
  \end{enumerate}
\end{lemma}
\begin{proof}
  \begin{enumerate}
    \item Assume $\mu$ is $\sigma-$additive, let $E\in \mathscr{C}$, and let $(E_n)_{n\geq 1}\in \mathscr{C}$ such that $E_n \uparrow E$. Let $F_1 = E_1$ and $F_n = E_n-E_{n-1}$. Clearly by this definition $\bigcup_{j\geq 1} F_j = \bigcup_{j\geq 1} E_j = E$. Then,
      \[\mu\l(\bigcup_{j\geq 1} F_j\r) = \sum_{j\geq 1}\mu(F_j) = \lim_{n\to\infty}\sum_{j\geq 2}^n\l(\mu(E_j)- \mu(E_{j-1})\r) + \mu(E_1) = \lim\mu(E_n)\]
      Hence $\mu$ is continuous from below. 
      \paragraph{} For proving continuity from above, let $(E_n)_{n\geq 1}\in \mathscr{C}$ such that some $\mu(E_{n_0}) <\infty$ and $E_n \downarrow E$. Let $G_m = E_{n_0} - E_{n_0+m}$ be a sequence of sets, $\bigcup_{m\geq n_0} G_m = E_{n_0} - E$. Using the fact the $\mu$ is continuous from below,
      \[\lim_{m\to\infty}\mu(G_m) = \mu(E_{n_0}) - \mu(E)\]
      hence,
      \[\lim_{m\to\infty}\mu(E_{n_0}) - \lim_{m\to\infty}\mu(E_{n_0+m}) = \mu(E_{n_0}) - \mu(E)\]
      \[\lim_{m\to\infty}\mu(E_{n_0+m}) = \mu(E)\]
      This is the same as $\lim \mu(E_n) = \mu(E)$.
    \item Assume that $\mu$ is continuous from below. Let $E\in \mathscr{C}$ be represented as the union of pairwise disjoint sets $E_1, E_2,...$. Let $F_1,F_2,...$ be a sequence defined as:
      \[F_k = \bigcup_{j=1}^{k} E_j\]
      Clearly $F_k$ is a sequence that conerges to $E$ from below. Using the fact that $\mu$ is additive and continuous from below:
      \[\mu(E) = \lim_{n\to\infty}\mu(F_n) = \lim_{n\to\infty} \mu\l(\bigcup_{j=1}^n E_j\r) = \lim_{n\to\infty}\sum_{j=1}^{n}\mu(E_j) = \sum_{j\geq 1}\mu(E_j)\]
      Hence $\mu$ is $\sigma-$additive.
    \item Assume that $\mu$ is continuous from above at $\emptyset$. Let $A\in \mathscr{C}$ and let $A_1, A_2,...$ be pairwise disjoints sets whose union is $A$. Define the sets $E_1, E_2,...$ as
      \[E_n = A - \bigcup_{j=1}^nA_j\]
      Clearly $E_n\downarrow \emptyset$. Using finiteness, additivity, and continuity from above at $\emptyset$,
      \begin{align*}
        \lim_{n\to\infty}\mu(E_n) &= 0\\
        \implies \lim_{n\to\infty} \mu\l(A - \bigcup_{j=1}^nA_j\r) &= 0\\
        \implies \mu(A) = \sum_{j\geq 1} \mu(A_j)
      \end{align*}
  \end{enumerate}
  This completes the proof.
\end{proof}
\begin{theorem}[Extension Theorem]\label{thm:extension}
  Let $ \mathscr{S}$ be a semi-algebra, $\mu: \mathscr{S}\to \R^+$ be an additive measure, and let $ \mathscr{A}$ be the algebra generated by $ \mathscr{S}$. Then there exists a $\nu: \mathscr{A}\to \R^+$, called the \textit{extension} of $\mu$, such that:
  \begin{enumerate}
    \item $\nu(A) = \mu(A),\ \forall\ A\in \mathscr{S}$.
    \item $\nu$ is additive.
  \end{enumerate}
  In addition such a measure on $ \mathscr{A}$ is unique.  
\end{theorem}
\begin{proof}
  Let $\nu: \mathscr{A}\to \R^+$ be a function defined in the following way. Using \cref{lem:generated} we know that for any $A\in \mathscr{A}$ we can find disjoint $E_1,...,E_n\in \mathscr{S}$ such that $A = \bigcup_{j=1}^n E_n$; then define $\nu$ as,
  \[\nu(A) = \sum_{j=1}^n\mu(E_j)\]
  First we must show that $\nu$ is well defined, since there can be more than one sequence of pairwise disjoint sets whose union is $A$. Let $E_1,...,E_n$ and $F_1,...,F_m$ be two sequences of pairwise disjoint sets in $ \mathscr{S}$ whose union is $A$. Then, 
  \[\nu(A) = \sum_{j=1}^n\mu(E_j) \]
  and
  \[\nu(A) = \sum_{j=1}^m\mu(F_j). \]
  Since $A = \bigcup_{k=1}^m F_k$
  \[\implies E_j = \bigcup_{k=1}^m F_k\cap E_j\]
  \[\implies \mu(E_j) = \sum_{k=1}^m \mu(F_k \cap E_j)\]
  Hence,
  \[\nu(A) =\sum_{j=1}^n\sum_{k=1}^m \mu(F_k \cap E_j) \]
  Similarly it can shown that,
  \[\mu(F_j) = \sum_{k=1}^n \mu(E_k \cap F_j)\]
  and therefore
  \[\sum_{j=1}^m\mu(F_j) =\sum_{j=1}^n\mu(E_j). \]
  Hence $\nu$ is well defined. \\

  Clearly for $A\in \mathscr{S}$ we have $\nu(A) = \mu(A)$. For additivity, let $A_1, A_2,...,A_n$ be pairwise disjoint sets in $ \mathscr{A}$ whose union is $A$. Again from \cref{lem:generated} for each $A_j = \bigcup_{k=1}^{n_j}E_{jk}$ where $E_{j1},...,E_{jn_j} \in \mathscr{S}$ are pairwise disjoint. Let $F_1,...,F_{N}$, where $N=\sum_{j=1}^n n_j$, be defined as $F_1 = E_{11}, F_2 = E_{12}$ and so on. Then,
  \begin{align*}
    A &= \bigcup_{j=1}^N F_j\\
    \implies \nu(A) &= \sum_{j=1}^{N} \mu(F_j) = \sum_{j=1}^{n}\sum_{k=1}^{n_j} \mu(E_{jk})
  \end{align*}
  since
  \begin{align*}
    \nu(A_j) &= \sum_{j=1}^{n_j} \mu(E_{jk}),\\
    \implies \nu(A) &= \sum_{j=1}^n \nu(A_j)
  \end{align*}
  Therefore $\nu$ is additive.\\

  For uniqueness, let's assume that two such functions $\nu_1$ and $\nu_2$ exist. From property $1$ we know that $\nu_1(A) = \nu_2(A) \forall\ A\in \mathscr{S}$. Let $A\in \mathscr{A}$ and let $A_1,...,A_n\in \mathscr{S}$ be pairwise disjoint with union $A$. Then using additivity 
  \[\nu_1(A) = \sum_{j=1}^n\nu_1(A_j) = \sum_{j=1}^n\nu_2(A_j) = \nu_2(A)\]
  This completes the proof.
\end{proof}
\begin{remark}
  This is theorem can be easily generalised for $\sigma-$additive measures. The only change in the proof would be considering countably many $A_j$ in the proof of additivity.
\end{remark}
