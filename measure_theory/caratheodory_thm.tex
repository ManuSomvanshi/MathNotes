\section{Caratheodory Theorem}
 Until now we have shown that extension $\nu$ of $\sigma-$additive measure $\mu$ on semi-algebra $ \mathscr{S}$ is also $\sigma-$additive on algebra $ \mathscr{A}$ generated by $ \mathscr{S}$. The goal of this section is to show that the extension $\pi: \mathscr{F}\to \R^+$, where $ \mathscr{F}$ is the $\sigma-$algebra generated by $ \mathscr{S}$ is $\sigma-$additive and unique. In order to do this we follow the following steps:
 \begin{enumerate}
   \item Define a $\pi^*: \mathfrak{P} (\Omega) \to \R^+$ and show that it is something called an \textit{outer measure}.
   \item Define a class $ \mathscr{M} \subset \mathfrak{P} (\Omega)$, and show that it is a $\sigma-$algebra.
   \item Show that $ \mathscr{A} \subset \mathscr{M}$. This has the implication that $ \mathscr{F} \subset \mathscr{M}$.
   \item Show that $\pi^*|_{ \mathscr{M} }$ is $\sigma-$additive and $\pi^*|_{ \mathscr{A} } = \nu$. Hence $\pi^*|_{ \mathscr{M} }$ is an extension.
   \item Finally show that this extension is unique.
 \end{enumerate}
 \begin{definition}
   Let $A \subset \Omega$ for some set $\Omega$. Then the collection $\{E_i\subset \Omega\ |\ i\geq 1\}$ is said to be a covering of $A$ if $A\subset \bigcup_{j\geq 1} E_j$. Note that at least one covering exists for every subset and that is $\{\Omega\}$.
 \end{definition}
 \begin{definition}
   Let $\pi^*: \mathfrak{P} (\Omega) \to \R^+$ for some set $\Omega$ defined in the following way: let $A \subset \Omega$ and let $\{E_i\in \mathscr{A}\ |\ i\geq 1\}$ be a covering of $A$ then
   \[\pi^*(A) = \inf_{\{E_i\}} \sum_{i\geq 1} \nu(E_i)\]
   This is to be read as infimum of $\sum_{i\geq 1} \nu(E_1)$ over all coverings of $A$ which are in the algebra $ \mathscr{A}$. 
 \end{definition}
 \begin{definition}
   Let $ \mathscr{C}$ be a class of subsets of $\Omega$ such that $\emptyset \in \mathscr{C}$, and let $\mu: \mathscr{C} \to \R^+$ be a function such that:
   \begin{enumerate}
     \item $\mu(\emptyset) = 0$,
     \item $\mu$ is monotone, i.e. $E\subset F$ where $E,F\in \mathscr{C}$ $\implies \mu(E)\leq \mu(F)$,
     \item $\mu$ is sub-additive, i.e. $E \in \mathscr{C}$ and $\{E_i\in \mathscr{C}\ |\ i\geq 1\}$ is a covering of $E$ then $\mu(E) \leq \sum_{i\geq 1} \mu(E_i)$.
   \end{enumerate}
   Then $\mu$ is said to be an outer measure. 
 \end{definition}
 \begin{proposition}\label{pro:outer}
   The function $\pi^*$ as defined above is an outer measure.
 \end{proposition}
 \begin{proof}
   Since $\emptyset \subset \Omega$, and it is a subset of every possible covering, clearly for the covering $\{E_i = \emptyset\ |\ \forall\ i\geq 1\}$,
   \[\sum_{i\geq 1} \nu(E_i) = 0\]
   and hence $\pi^*(\emptyset) = 0$.\\

   Let $E\subset F$ where $E,F \in \mathscr{C}$. Let $\{F_j\ |\ j\geq 1\}$ be a covering of $F$. Since $E\subset F$ any covering of $F$ is also a covering of $E$. If $E_j = F\cap F_j$ then $E_j \subset F_j$ and $\bigcup_{j\geq 1}E_j = F \supset E$. Hence $\{E_j\}$ is a covering $E$. Since $\nu$ is a $\sigma-$additive measure,
   \[\nu(E_j) \leq \nu(F_j)\ \text{and hence,}\ \sum_{i\geq 1}\nu(E_i) \leq \sum_{i\geq 1}\nu(F_i).\]
   Since for every covering of $F$ a covering of $E$ can be constructed in the above manner such that the above inequality is true, hence $\pi^*$ is monotone. \\

   For sub-additivity let $E \subset \Omega$ and let $\{E_i \in \mathscr{A}\ |\ i\geq 1\}$ be a covering of $E$. In the case when $\pi^*(E_i) = \infty$, clearly $\pi^*(E) \leq \pi^*(E_i)$. In the case when $\pi^*(E_i) < \infty\ \forall\ i\geq 1$, for each $\epsilon>0$ we can find a covering of $E_i$, say $\{F_{ij}\in \mathscr{A}\ |\ j\geq 1\}$ such that,
   \[\pi^*(E_i) \leq \sum_{j\geq 1} \nu(F_{ij}) \leq \pi^*(E_i) + \f{\epsilon}{2^i}\]
   Hence,
   \[\sum_{i\geq 1}\pi^*(E_i) \leq \sum_{j\geq 1} \nu(\bigcup_{j\geq 1}F_{ij}) = \sum_{i\geq 1}\nu(E_i)\leq \sum_{i\geq 1}\pi^*(E_i) + \epsilon\]
   Using $\pi^*(E) \leq \sum_{i\geq 1}\nu(E_i)$, we get
   \[\pi^*(E) \leq \sum_{i\geq 1}\nu(E_i) \leq \sum_{i\geq 1}\pi^*(E_i) + \epsilon\]
   Since this is true for arbitrary $\epsilon$ we conclude that $\pi^*$ is sub-additive. 
 \end{proof}
 \begin{definition}
   A set $A\subset\Omega$ is said to be measurable if $\forall\ E\subset \Omega$,
   \[\pi^*(E) = \pi^*(E\cap A) + \pi^*(E\cap A^c)\]
   Define $ \mathscr{M} $ to be the set of all measurable subsets of $ \Omega$.
 \end{definition}
 \begin{remark}
   Using sub-additivity of $\pi^*$ it is possible to prove that
   \[\pi^*(E) \leq \pi^*(E\cap A) + \pi^*(E\cap A^c)\]
   since $E = (E\cap A)\cup (E\cap A^c)$. Hence showing that $A$ is measurable just boils down to showing
   \[\pi^*(E) \geq \pi^*(E\cap A) + \pi^*(E\cap A^c)\]
 \end{remark}
 \begin{proposition}
   Any alegebra $ \mathscr{B}$ of subsets of $\Omega$ is a subset of $ \mathscr{M}$.
 \end{proposition}
 \begin{proof}
   Let $A\in \mathscr{B}$ and let $E\in\Omega$. If we can show that $A$ is measurable, we prove the proposition. Let $\{E_i \in \mathscr{A}\}$ be a covering of $E$, and let $\e>0$. In the case when $\pi^*(E_i) = \infty$ even for a single $i$, it is clear that the inequality
   \begin{align}
     \pi^*(E) \geq \pi^*(E\cap A) + \pi^*(E\cap A^c)
   \end{align}
   holds. In the case when $\pi^*(E_i) < \infty$ forall $i\geq 1$ let $\e>0$. Then
   \[\pi^*(E) \leq \sum_{i\geq 1}\nu(E_i) \leq \pi^*(E) + \e \]
   Since $E\cap A \subset \bigcup_{i\geq 1} E_i\cap A$,
   \[\pi^*(E\cap A) \leq \sum_{i \geq 1} \nu(E_i \cap A)\]
   Using similar arguments for $A^c$
   \[\pi^*(E\cap A^c) \leq \sum_{i \geq 1} \nu(E_i \cap A^c)\]
   Adding these two inequalities
   \[\pi^*(E\cap A^c) + \pi^*(E\cap A) \leq \sum_{i \geq 1} \nu(E_i)\]
   Here I have used the additivity of $\nu$ since $E_i\cap A$ and $E_i\cap A^c$ are in the algebra. Using inequality (1):
   \[\pi^*(E\cap A^c) + \pi^*(E\cap A) \leq \sum_{i \geq 1} \nu(E_i) \leq \pi^*(E) + \e\]
   Since this is true for arbitrary $\e$, we have
   \[\pi^*(E\cap A^c) + \pi^*(E\cap A)  \leq \pi^*(E) \]
   Hence we have shown that $A$ is measurable, completing the proof.
 \end{proof}
 \begin{proposition}
   $ \mathscr{M}$ is a $\sigma-$algebra.
 \end{proposition}
 \begin{proof}
   Since every algebra is a subset of $ \mathscr{M}$ clearly $\Omega \in \mathscr{M}$. Also it is easy to see that if $A\in \mathscr{M}$ then $A^c \in \mathscr{M}$ since replcaing $A$ by $A^c$ in the condition of measurable set does not change the inequality.\\
   The only condition that remains to be checked is closure under countable union. 
 \end{proof}
