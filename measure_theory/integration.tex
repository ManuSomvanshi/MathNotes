\section{Integration}
In this section we would like to formulate the concept of integral in the context of measure spaces. In the Riemann integral we first partition the domain and then approximate then integral to be:
\[\int f \approx \sum_{k\geq 1} y_k (x_k - x_{k-1})\]
Using a similar concept we would later define an integral, called the Lebesgue integral, where we partition the $y$-axis, take the inverse of that interval to get a set in the domain, and then use a measure to find the "length" of this interval. Then we approximate the integral as follows:
\[\int f \approx \sum_{k\geq 1} y_k \mu(f^{-1}(A_k))\]
See \cref{fig:int} to understand this better. But in order to define this integral the function must have the property that its inverse belongs to the $\sigma-$algebra on which the measure $\mu$ is defined. For this a new class of functions known as measurable functions is defined which hold this property. 
\begin{figure}
  \centering
  \begin{subfigure}[b]{0.45\textwidth}
    \incfig[0.7]{riemann_int}
    \caption{Riemann integral}
  \end{subfigure}
  \hfill
  \begin{subfigure}[b]{0.45\textwidth}
    \incfig[0.7]{lebesgue_int}
    \caption{Lebesgue integral}
  \end{subfigure}
  \caption{Visualization of the integral}
  \label{fig:int}
\end{figure}
\begin{definition}
  Let $(X,\mathscr{T})$ be a topological space. Then the $\sigma-$algebra generated by the open sets of this space is called the \textit{Borel $\sigma-$algebra}, and represented $ \mathscr{B}(X,\mathscr{T})$.\\

  In the case when $X=\R^n$ and $\mathscr{T}$ is the usual topology on $\R$, the Borel $\sigma-$algebra is simply denoted $ \mathscr{B}( \R^n)$.
\end{definition}
\begin{definition}
  Let $(\Omega, \mathscr{F}, \mu)$ be a measure space, and $f:\Omega \to \R$ be a function $\Omega$ then $f$ is said to be $ \mathscr{F}-$\textit{measurable}, or simply measurable, if $B\in \mathscr{B}( \R) \implies f^{-1}(B)\in \mathscr{F}$.\\

  In general if $(\Omega_1, \mathscr{F}_1, \mu_1)$ and $(\Omega_2, \mathscr{F}_2, \mu_2)$ are measure spaces then $f: \Omega_1 \to \Omega_2$ is said to be $ \langle\mathscr{F}_1, \mathscr{F}_2 \rangle$-measurable if $A\in \mathscr{F}_2 \implies f^{-1}(A) \in \mathscr{F}_1$. Hence $ \mathscr{F}-$measurable functions are just $\langle \mathscr{F}, \mathscr{B}( \R)\rangle$-measurable.
\end{definition}
\begin{lemma}\label{lem:measurable_f}
  Let $(\Omega, \mathscr{F}, \mu)$ be a measure space and $f:\Omega \to \R$. Then $f$ is measurable $\iff\ f^{-1}((-\infty, x]) \in \mathscr{F}$.
\end{lemma}
\begin{proof}
  Before we begin proving this lemma, note that $ \mathscr{B}(\R)$ is the $\sigma-$algebra generated by the usual topology $\mathscr{T}$ on $\R$. Let $ \mathscr{C}=\{(-\infty,x]\ |\ x<\infty\}$ and $ \mathscr{G}$ be the $\sigma-$algebra generated by $ \mathscr{C}$. Since for any $(-\infty, x]$ the sequence of elements $(-\infty, x_n) \in \mathscr{B}(\R)$ where $x_n \downarrow x$ satisfies $(-\infty,x] = \bigcap_{j\geq 1}(-\infty, x_j)$. This shows that $(-\infty, x]\in \mathscr{B}(\R)$ (using closure under countable intersection), furhter implying that $ \mathscr{G}\subset \mathscr{B}(\R)$. Similarly it can be shown that any open set $(x,y)$ can be expressed as countable unions and intersections of elements of $ \mathscr{G}$, implying that $ \mathscr{B}(\R)\subset \mathscr{G}$. Hence $ \mathscr{B}(\R) = \mathscr{G}$. 
  \paragraph{($\implies $)} If $f$ is measurable then $A\in\mathscr{B}( \R)\implies f^{-1}(A) \in \mathscr{F}$, and since $(-\infty, x]\in \mathscr{B}( \R)$ this implies that $f^{-1}((-\infty,x])\in \mathscr{F}$.   
  \paragraph{($\impliedby $)} Let $ \mathscr{M} = \{A\in \mathscr{B}(\R)\ |\ f^{-1}(A)\in \mathscr{F}\}$. Since $\R \in \mathscr{B}(\R)$ and $f^{-1}(\R) = \{\omega\ |\ f(\omega)\in \R \} = \Omega \implies \Omega \in \mathscr{M}$. Also observing that $f^{-1}$ preserves countable unions and compliments, if $A, A_1,... \in \mathscr{M}$ then $f^{-1}(A^c) = (f^{-1}(A))^c\in \mathscr{F}$ and $f^{-1}(\bigcup_{j\geq1}A_j) = \bigcup_{j\geq 1}f^{-1}(A_j)$. This shows that $ \mathscr{M}$ is closed under compliments and countable unions. Hence $ \mathscr{M}$ is a $\sigma-$algebra. Since we are showing the backward implication we assume that $ \mathscr{C} \subset \mathscr{M}$. Since $ \mathscr{M}$ is a $\sigma-$algebra containing $ \mathscr{C}$ it follows that $ \mathscr{B}(\R) \subset \mathscr{M}$. Since by definition of $ \mathscr{M}$ it is a subset of $ \mathscr{B}(\R)$ further follows that $ \mathscr{M} = \mathscr{B}(\R)$. Thus proving the lemma.
\end{proof}
\begin{remark}
  Note that the key to proving the lemma really was showing that $ \mathscr{B} (\R) \subset \mathscr{M}$, which required the fact that $ \mathscr{B}(\R)$ was the $\sigma-$algebra generated by $ \mathscr{C}$. Hence this theorem can be easily extended to any class $ \mathscr{C}$ which generates the $\sigma-$algebra $ \mathscr{B}(\R)$, in the following sense:
  \begin{lemma}
  Let $(\Omega, \mathscr{F}, \mu)$ be a measure space, $ \mathscr{C}$ be a class of subsets of $\R$ which generates the $\sigma-$algebra $ \mathscr{B} (\R)$, and $f:\Omega \to \R$. Then $f$ is measurable $\iff\ f^{-1}(A) \in \mathscr{F}$ where $A\in \mathscr{C}$.
  \end{lemma}
  Hence \cref{lem:measurable_f} also holds if $(-\infty, x]$ is replaced with one of $(-\infty, x),\ (x, \infty)$, or $[x,\infty)$. 
\end{remark}
\begin{definition}
  Let $(\Omega, \mathscr{F}, \mu)$ be a measure space, $E_1,...,E_n\in \mathscr{F}$ be pairwise disjoint sets such that $\Omega = \bigcup_{j=1}^n E_j$, and $1_{E_j}$ be the indicator function of $E_j$. Then a \textit{simple function} $f:\Omega \to \R$ is a function which can be written as:
  \[f(\omega) = \sum_{j=1}^n c_j 1_{E_j}(\omega)\]
  where $c_j\in \R$.
\end{definition}
\begin{proposition}
  Simple functions are measurable. 
\end{proposition}
\begin{proof}
  Let $f:\Omega\to \R$ be a simple function expressed as:
  \[f(\omega) = \sum_{j=1}^n c_j 1_{E_j}(\omega)\]
  The set $f^{-1}((-\infty,x]) = \{\omega\ |\ f(\omega) \leq x\}$ should belong to $ \mathscr{F}$ for $f$ to be measurable by \cref{lem:measurable_f}. Notice that $f(\omega)\leq x$ only when $\omega\in E_j$ such that the corresponding $c_j \leq x$. Hence $f^{-1}((-\infty, x]) = \bigcup_{j\ |\ c_j\leq x} E_j$. Since each $E_j\in \mathscr{F}$ so will be any finite union. Hence $f^{-1}((-\infty, x])\in \mathscr{F}$.
\end{proof}
\begin{definition}[Integral of Non-negative Simple functions]
  Let $(\Omega, \mathscr{F}, \mu)$ be a measure space and let $f$ be a non-negative simple function of the form 
  \[f(\omega) = \sum_{j=1}^n c_j 1_{E_j}(\omega),\ c_j\leq 0\]
  Then we define:
  \[\int f = \sum_{j=1}^n c_j \mu(E_j)\]
\end{definition}
The non-negative condition was applied to avoid cases like the following: $\mu(E_1)= \mu(E_2)= \infty$ and $c_1 = -c_2$. Then the sum on the RHS would have $\infty - \infty$, which is not well defined.  
\begin{proposition}
  The integral of non-negative simple function is well defined.
\end{proposition}
\begin{proof}
  Let $f:\Omega\to\R$, $\{E_1,...,E_n\}, \{F_1,...,F_2\}\in \mathscr{F}$ be two partitions of $\Omega$, and $f$ be represented as:
  \[f(\omega) = \sum_{j=1}^n c_k 1_{E_j} = \sum_{j=1}^n d_k 1_{F_j}\]
  where $c_j,d_j\geq 0$. Consider the case when $E_{j_0}\cap F_{k_0} \neq \emptyset$. If $\omega \in E_{j_0}\cap F_{k_0}$ then $f(\omega) = c_{j_0} = d_{k_0}$. Since
  \begin{align*}
    \mu(E_j) &= \mu(E_j \cap \Omega)\\
          &= \mu(E_j \cap \bigcup_{k=1}^n F_k)\\
          &= \sum_{k=1}^n \mu(E_j\cap F_k),
  \end{align*}
  it follows that
  \begin{align*}
    \int f = \sum_{j=1}^n \sum_{k=1}^n c_j \mu(E_j\cap F_k).
  \end{align*}
  Similarly in case of $F_j$,
  \begin{align*}
    \int f = \sum_{j=1}^n \sum_{k=1}^n d_k \mu(E_j\cap F_k).
  \end{align*}
  When $E_j\cap F_k = \emptyset$ the corresponding term in the sum is $0$, and when $E_j\cap F_k \neq \emptyset$ then $d_j=c_j$. Thus both the sums are equal. 
\end{proof}
\begin{lemma}
  Let $(\Omega, \mathscr{F}, \mu)$ be a measure space, $f,g: \Omega \to \R$ be measurable functions, and $\alpha$ be some constant; then
  \begin{enumerate}
    \item $\alpha f$, 
    \item $f + \alpha$, 
    \item $f+g$,
    \item $f^2$,
    \item $1/f$,
    \item $f^\pm$, where $f^\pm(\omega) = \max(\pm f(\omega), 0)$,
    \item $fg$
  \end{enumerate}
  are measurable functions.
\end{lemma}
\begin{proof}
  From \cref{lem:measurable_f}, in each case we only have to show that the inverse map of $(-\infty,x]$ belongs to $ \mathscr{F}$, given that $A\in \mathscr{B}( \R)\implies f^{-1}(A),g^{-1}(A)\in \mathscr{F}$. 
  \begin{enumerate}
    \item In this case we need to show that $\{\omega\ |\ \alpha f(\omega)\leq x\}\in \mathscr{F}$. When $\alpha = 0$, forall $x\geq 0$ the set in question is $\Omega$ and when $x<0$ it is $\emptyset$. Both of these are in $ \mathscr{F}$. When $\alpha>0$, the set $\{\omega\ |\ \alpha f(\omega) \leq x\} = \{\omega\ |\ f(\omega)\leq x/\alpha\} \in \mathscr{F}$. Similarly for $\alpha < 0$, $\{\omega\ |\ \alpha f(\omega)\leq x\} = \{\omega\ |\ f(\omega)\geq x/\alpha\}\in \mathscr{F}$.\\
    \item Using similar logic as above $\{\omega\ |\ -\infty <f(\omega) +\alpha \leq x\} =\{\omega\ |\ -\infty <f(\omega) \leq x-\alpha \} \in \mathscr{F}$. \\
    \item Consider the set $\{\omega\ |\ f(\omega)+g(\omega) \leq x\}$. Using density of $ \mathbb{Q}$ in $\R$ we know that it is always possible to find $r\in \mathbb{Q}$ such that $f(\omega)\leq r$ and hence $g(\omega)\leq x-r $. Thus $\{\omega\ |\ f(\omega)+g(\omega) \leq x\} = \bigcup_{r\in \mathbb{Q}}\{\omega\ |\ f(\omega) \leq r\}\cap \{\omega\ |\ g(\omega) \leq x-r\} $. Since each $\{\omega\ |\ f(\omega) \leq r\}$ and $\{\omega\ |\ g(\omega) \leq x-r\}$ is in $ \mathscr{F}$, by closure under countable unions and intersections $\{\omega\ |\ f(\omega)+g(\omega) \leq x\}\in \mathscr{F}$.\\
    \item Consider the set $\{\omega\ |\ f^2(\omega) \leq x\}$. In the case when $x<0$, the set $\{\omega\ |\ f^(\omega) \leq x\} = \emptyset\in \mathscr{F}$. In the case when $x\geq0$, $\{\omega\ |\ f^2(\omega) \leq x\} = \{\omega\ |\ -\sqrt{x} \leq f(\omega) \leq \sqrt{x}\}\in \mathscr{F}.$\\   
    \item Consider the set $\{\omega\ |\ 1/f(\omega) < x\}$. In the case when $x>0$, 
      \begin{align*}
        \{\omega\ |\ 1/f(\omega) < x\} &= \{\omega\ |\ 1/f(\omega) < 0\}\cup \{\omega\ |\ 0<1/f(\omega) < x\}\\ &= \{\omega\ |\ f(\omega) \leq 0\}\cup \{\omega\ |\ 0<f(\omega) \leq 1/x\}
      \end{align*}
      Since each set in the RHS is in $ \mathscr{F}$ it follows that $\{\omega\ |\ 1/f(\omega) < x\}\in \mathscr{F}$. When $x=0$, $\{\omega\ |\ 1/f(\omega) < 0\} = \{\omega\ |\ f(\omega) < 0\} \in \mathscr{F}$. When $x<0$ then
      \[\{\omega\ |\ 1/f(\omega) < x\} = \{\omega\ |\ 0>f(\omega) > 1/x\} \]
      which is clearly in $ \mathscr{F}$.\\
    \item In the case of product of measurable functions, we simply use the identity:
      \[fg = \f{1}{2} \big((f+g)^2 - f^2 - g^2\big)\]
      Since $f+g$, $f^2$, and $g^2$ are measurable (by points 3,4) it follows that $fg$ is also measurable (again by point 3).
  \end{enumerate} 
\end{proof}
